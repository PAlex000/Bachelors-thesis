\documentclass[12pt,a4paper]{report}
\usepackage[margin=2.5cm]{geometry}
\usepackage[magyar]{babel}

% magyar nyelv tamogatas
\usepackage{t1enc}
\usepackage[T1]{fontenc}
\usepackage[utf8]{inputenc}

% A formai kovetelmenyekben megkövetelt Times betűtípus hasznalata:
\usepackage{times}

\usepackage{setspace}
\usepackage{listings,multicol}
\usepackage{xcolor}
\usepackage{textcomp}
\usepackage{enumitem}
\usepackage{booktabs}
\usepackage[unicode,hidelinks]{hyperref}
\usepackage{footnote}
\usepackage{ifthen}

% Törölhető package
\usepackage{lipsum}

% TODO csomag, amivel jól észrevehető todokat hagyhatunk a dolgozatban
\usepackage{todonotes}
% Inline TODO
\newcommand{\todoi}[1]{\todo[inline]{\textbf{TODO:} #1}}

% egyedi lablec
\usepackage{fancyhdr}
\usepackage{graphicx}
\graphicspath{{fig/}}

% Kódrészletes színezése
\definecolor{lightgray}{rgb}{.9,.9,.9}
\definecolor{darkgray}{rgb}{.4,.4,.4}
\definecolor{purple}{rgb}{0.65, 0.12, 0.82}

\lstdefinelanguage{JavaScript}{
  keywords={typeof, new, true, false, catch, function, return, null, catch, switch, var, if, in, while, do, else, case, break},
  keywordstyle=\color{blue}\bfseries,
  ndkeywords={class, export, boolean, throw, implements, import, this},
  ndkeywordstyle=\color{darkgray}\bfseries,
  identifierstyle=\color{black},
  sensitive=false,
  comment=[l]{//},
  morecomment=[s]{/*}{*/},
  commentstyle=\color{purple}\ttfamily,
  stringstyle=\color{red}\ttfamily,
  morestring=[b]',
  morestring=[b]"
}

\lstset{
   language=JavaScript,
   backgroundcolor=\color{lightgray},
   extendedchars=true,
   basicstyle=\footnotesize\ttfamily,
   showstringspaces=false,
   showspaces=false,
   numbers=left,
   numberstyle=\footnotesize,
   numbersep=9pt,
   tabsize=2,
   breaklines=true,
   showtabs=false,
   captionpos=b
}

\renewcommand{\lstlistingname}{Kódrészlet}

% Margók beállítása
\hoffset -1in
\voffset -1in
\oddsidemargin 35mm
\textwidth 150mm
\topmargin 15mm
\headheight 10mm
\headsep 5mm
\textheight 237mm

% Szerző és dolgozat adatai
%Szerző adatai
\newcommand{\nev}{Pozsgai Alex}
\newcommand{\szak}{programtervező informatikus BSc}
\newcommand{\tanszek}{Szoftverfejlesztés}
\newcommand{\ev}{2023}
\newcommand{\dolgozatTipusa}{Szakdolgozat}
\newcommand{\vegsoDatum}{\today}

\newcommand{\cim}{Ipari JavaScript elemző kiegészítése TypeScript támogatással}

%Témavezető adatai
\newcommand{\temavezetoNev}{Dr. Antal Gábor}
\newcommand{\temavezetoBeosztas}{Tudományos munkatárs}


\begin{document}

% Másfeles sorköz
\setstretch{1.5}
\sloppy

\pagestyle{fancy}
\fancyhf{}
\fancyhead[L]{\textit{\cim}}
\fancyfoot[R]{\thepage}
\fancypagestyle{plain}{%
    \renewcommand{\headrulewidth}{0pt}%
    \fancyhf{}%
    \fancyfoot[R]{\thepage}%
}

\include{tex/0_Title}

\pagenumbering{arabic}
\chapter*{Feladatkiírás}
\addcontentsline{toc}{section}{Feladatkiírás}

\noindent

Manapság kezd a piacon egyre több kódelemező megjelenni, többek között JavaScript programozási nyelvre is. Jelenlegi projekt is egy JavaScript Analyzer. 
A cél az, hogy a projekt több legyen, mint a piacon a többi kódelemező, ezért bővíteni kell TypeScript támogatással.

\noindent

A hallgató feladata ennek a programnak(JSAN) a fejlesztése úgy, hogy tudjon TypeScript fileokat és projekteket is kellő pontossággal elemezni. 
Ezután pedig a meglévő programot optimalizálni, hogy kevesebb erőforrást vegyen igénybe a futtatása, és hogy gyorsabban fusson le.

\noindent

A megoldási módszerek a hallgató kreativitására vannak bízva.

\chapter*{Tartalmi összefoglaló}
\addcontentsline{toc}{section}{Tartalmi összefoglaló}

\noindent\textbf{A téma megnevezése:}

JavaScript Analyzer kiegészítése TypeScript supporttal, JSAN optimalizálása.

\noindent\textbf{A megadott feladat megfogalmazása:}

A feladat során el kell érni azt, hogy a JavaScript Analyzer Tool tudjon TypeScript fileokra lefutni, és azokat nagy pontossággal elemezni. 
Emellett a JSAN működését optimalizálni.

\noindent\textbf{A megoldási mód:}

A megoldás során kell változtatni a JavaScriptSchémán, és a JSAN-ban az AstTransformer.js fileban.

\noindent\textbf{Alkalmazott eszközök, módszerek:}

A megoldáshoz Visual Studio Code IDE-t, a JavaScriptSchema szerkesztéséhez Visual paradigm-t használta.
A projekt lebuildeléséhez és ellenőrzéséhez Visual Studio 2017 programot használtam.

\noindent\textbf{Elért eredmények:}

JSAN képes TypeScript fileokat nagy pontossággal elemezni, nagyobb projektre lényegesen gyorsabban fut le, kevesebb erőforrást igényel, mint előtte.

\noindent\textbf{Kulcsszavak:}

JavaScriptAnalyzer, TypeScriptAnalyzer, JavaScript, TypeScript, Optimalizálás, Visual Studio Code, Visual Paradigm, Vpp, C++

\thispagestyle{plain}
\tableofcontents


\chapter{Bevezetés}

\noindent

Szakdolgozatomban a SourceMeter for JavaScript továbbfejlesztése volt a cél.
A JavaScript elemzése mellett TypeScript nyelvű fájlokat és projekteket is elemezni kellett.
Ez azt eredményezte, hogy gyökerestül át kellett írni, annak érdekében,
hogy JavaScriptet és TypeScriptet is tudjon egyaránt elemezni.

\noindent

A SourceMeter for JavaScript projekten többen is fejlesztettek egyszerre, ezáltal voltak átfedések, oszthatatlan részek.
A projekt több része is egymásra épül, ezért kiemelten fontos volt a csapatmunka egyes részeknél, hogy a projekt eredményesen és hatékonyan haladjon.
Az eredményes csapatmunka magában foglalja az eredmények és az előrehaladás folyamatos kommunikálását,
ami azt jelenti, hogy mindketten megértjük, hogy milyen célkitűzések vannak a projektben,
és rendszeresen jelentést adunk egymásnak az elvégzett munkáról és az elért eredményekről.

\noindent

Megtalálható számtalan JavaScript és TypeScript fájl elemző eszköz amiknek nyílt a forráskódja.
Ezek között megtalálható a Codehawk CLI, Codelyzer vagy a CodeClimmate-Duplication.
Mind a három eszköz jól tud elemezni JavaScript és TypeScript fájlokat, viszont csak egy-egy specifikus esetre jók.

\noindent

Fejlesztés során a csoportos munka elkerülhetetlen volt, hiszen volt olyan rész, amire számtalan eszköz épült.
Ez a JavaScript nyelvi séma (továbbiakban:séma) szerkesztése volt. Ezt \aref{chap:nyelvi_sema} alfejezetben taglalom bővebben.

\noindent

A séma szerkesztése után a JavaScript Analyzer To Lim (továbbiakban JSAN2Lim) továbbfejlesztése volt a feladatom.
A továbbfejlesztés C++, C nyelv és TypeScript séma$^{~\cite{typescript-eslint}}$ ismeretét igényelte meg.
Emiatt a fejlesztés előtt tanulmányoznom kellett ezeket.

\noindent

A JSAN2Lim fejlesztése után a JavaScript elemzőnek (JavaScript Analyzer, továbbiakban: JSAN) a binder függvényét kellett optimalizálnom.
Először értelmeznem kellett a kódot, át kellett néznem az AST-t, illetve több adatszerkezetet is, gyorsasági szempontból.

\noindent

Legvégül bővítettem a regressziós teszteket új projektek behozatalával. Ezt \aref{chap:Regteszt_frissítés} fejezetben taglalom.
% Ezeknek az outputjait megnéztem, hogy jó eredményt ad-e, itt még volt egy kis bugfixing a JSAN2Limben és a Schémában.

\chapter{SourceMeter}\label{chap:SourceMeter}

\section{SourceMeter bevezetés}

\noindent

A SourceMeter egy forráskód-elemző eszköz, amely képes mély statikus programelemzést végezni a C, C++, Java, Python, C\#, JavaScript, TypeScript és RPG (AS/400)$^{~\cite{szHoke2014case}}$
nyelvű összetett programok forráskódján.
A FrontEndART a Szegedi Tudományegyetem Szoftverfejlesztés Tanszékén kutatott és fejlesztett Columbus technológián$^{~\cite{beszedes2005columbus}}$ alapuló SourceMeter eszközt fejlesztette ki.
A statikus kódelemzés egy olyan módszer, amely során a program forráskódját elemezzük, anélkül hogy azt ténylegesen futtatnánk.
Az elemzés során különböző eszközök segítségével ellenőrizhetjük a kód helyességét, hatékonyságát, biztonságosságát és karbantarthatóságát.
Az ilyen típusú elemzés során gyakran felhasználnak különböző szabályokat és előírásokat, amelyek segítenek az azonosításban és a hibák javításában.

\noindent

A statikus elemzés során absztrakt szemantikus gráf (ASG) készül a forráskód nyelvi elemeiből.
Ezután az ASG-t különböző eszközökkel dolgozzák fel a csomagban annak érdekében, hogy kiszámítsák a metrikákat (LLOC$^{~\cite{siket2014differences}}$, NLE vagy NOA),
azonosítsák az ismételt kódrészleteket (másolás-beszúrás; klónok), a kódolási szabályszegéseket, stb.
A SourceMeter képes elemzést végezni olyan forráskódon,amely megfelel a Java 8 és korábbi verzióinak, a C/C++,
az RPG III és az RPG IV verzióinak (beleértve a szabadon formázottakat), a C\# 6.0 és korábbi verzióinak, valamint a Python 2.7.8 és korábbi verzióinak.
A C/C++ esetében a SourceMeter támogatja az ISO/IEC 14882:2011$^{~\cite{sourcemeter2015}}$ nemzetközi szabványt, amelyet kiegészítettek az ISO/IEC 14882:2014 új funkcióival, és a C nyelvet az ANSI/ISO 9899:1990, az ISO/IEC 9899:1999 és az ISO/IEC 9899:2011 szabványok határozzák meg.
Az alapértelmezett funkciókon túl, a GCC és a Microsoft által meghatározott kiterjesztések is támogatottak.

\noindent

A SourceMeter a QualityGate eszközben van használva. \todoi{Képeket betenni}

\noindent

A SourceMeternek található egy plug-in a SonarQubehoz.
A SourceMeter plug-in a SonarQube platformhoz egy kiterjesztése az nyílt forráskódú SonarQube platformnak, amelyet a kód minőségének kezelésére használnak
A plug-in a SourceMeter-t futtatja a SonarQube platformról, és feltölti a forráskód elemzésének eredményeit a SourceMeter-től a SonarQube adatbázisába.
A plug-in nyílt forráskódú, és az összes szokásos SonarQube kódelemzési eredményt biztosítja, kiegészítve sok további metrikával és problémakeresővel, amelyeket a SourceMeter eszköz biztosít.
A plug-in támogatja a C/C++, a Java, a C\#, a Python és a RPG nyelveket.$^{~\cite{ferenc2014source}}$

\section{SourceMeter for JavaScript}

\noindent

A SourceMeter for JavaScript a SourceMeternek egy nagyobb alprojektje.
A SourceMeter for JavaScript egy olyan eszköz, amely lehetővé teszi a mély statikus forráskód elemzést a bonyolult JavaScript és TypeScript rendszerekben.
Képes felismerni a kód hibáit, mint például a nem definiált változók vagy függvények használata, a nem biztonságos kódrészletek, a nem hatékony kódrészletek, valamint a redundáns és ismétlődő kódok.
Ezenkívül az eszköz képes összehasonlítani a kódot az általános gyakorlatokkal és a meghatározott szabályokkal, és jelezni az eltéréseket.
Az ilyen eszközök használata segíthet az észlelt hibák javításában és a kód minőségének javításában, ami végső soron javíthatja a rendszer biztonságát és hatékonyságát.

\noindent

A SourceMeter for JavaScript projekt több alprojektet is magába foglal, amelyek különböző részfeladatokra specializálódnak.

\subsection{Nyelvi séma bevezetés}\label{chap:nyelvi_sema}
% Sajnos dokumentáció nem készült az előző filehoz, ezért nekünk kellett kitalálni, hogy mi mit csinált, hiszen akik ezt írták, ők már nem foglalkoztak ezzel és elérhetetlenek voltak.
% A szerkesztéshez szükséges volt a Visual Paradigm alkalmazást használni.
% Ezzel egyikőnk sem találkozott még, szóval először ezt kellett tanulmányozni, megérteni.
% Ezután értelmezni kellett a meglévő schemát, hiszen eddig az jól működött, csak nem lehetett könnyen bővíteni.
% Mérlegeltük a két opciót, ahol vagy megpróbáljuk bővíteni a jelenlegi schémát, vagy nulláról elkezdjük újraírni.
% Végül az újraírás mellett döntöttünk, hiszen ezt láttuk gyorsabb és könnyebb megoldásnak. Egy könnyen bővíthető, dokumentál schema volt az elképzelés.
% Ketten fejlesztettük le végül ezt a schemát.

\subsection{JavaScriptAddon bevezetés}

\subsection{JSAN bevezetés}

\subsection{JSAN2Lim bevezetés}

\subsection{Regressziós tesztelés bevezetés}

\chapter{Oszthatlan közösen készített programok}\label{chap:oszthatlan_kozos_dolgok}

\section{A nyelvi séma átírása}
Az eddigi JavaScript nyelvi séma (továbbiakban: séma), ami csak JavaScript nyelvű projekteket és fájlokat elemzett, a JavaScript hivatalos oldala~\cite{javascript_language} alapján készült.
\Aref{chap:nyelvi_sema}-es fejezetben ami ábrák láthatóak, azok már az átírt sémából lettek kifotózva.


A séma átírásánál arra a döntésre jutottunk, hogy elölről kezdjük az egészet.
Átláthatatlan volt az előző séma, és nem volt hozzá megfelelő dokumentáció, ezért a bővítés nehezebbnek bizonyult, mint az újraírás a kezdetekről.
Továbbá, egy séma ami képes TypeScript fájlokat és projekteket elemezni, az képes JavaScript fájlokat és projekteket is.
A Base csomagot emeltük át az előző sémából, BaseNode-al és BaseToken-el kibővítve. Ez látható \aref{fig:base_vpp}-es ábrán is.
A TypeScript-eslint github~\cite{typescript-eslint} alapján hoztuk létre a struktrális felépítést,
annyi változtatással, hogy mi hoztunk létre még egy structure nevű csomagot, amiben azok az osztályok találhatóak meg, amelyek a TypeScript-eslint github base mappájában voltak.
Fejlesztés során dokumentáltuk a lépések nagy részét.


Minden egymásra épült a sémában, emiatt a műkődést nem tudtuk az elején letesztelni, csak miután több csomag is készen lett.
Megfigyelhető \aref{lst:asg_file_export_all_declaration}-es kódrészleten, hogy az ExportAllDeclaration több mindenből öröklődik és sok attribútuma van.
Az ExportAllDeclaration egyik attribútumának a típusa az ImportAttribute.
Ehhez az ImportAttribute osztályt is le kellett fejleszteni a sémában.
\pagebreak
\begin{lstlisting}[caption={ImportAttribute},label={lst:asg_file_import_attribute}, language={JavaScript}]
export interface ImportAttribute extends BaseNode {
      type: AST_NODE_TYPES.ImportAttribute;
      key: Identifier | Literal;
      value: Literal;
}
\end{lstlisting}

Az ImportAttribute-hoz szükséges volt az Identifiert, és a Literalt lefejleszteni.
Ezekhez más-más osztályok lefejlesztése is szükséges volt.


\section{Nehézségek, problémák}
A séma újból írása során több nehézségbe ütköztünk.
Dokumentáció hiánya miatt nem tudtuk az előző sémának minden részét jól értelmezni, néhány félreértés volt fejlesztés közben.
Miután az összes csomaggal és osztállyal kész voltunk, próbáltuk tesztelni.
Tesztelés során Segmentation Fault hibát kapott a program, és nem tudtunk rájönni, hogy miért történik ez.
Emiatt az egész sémát átnéztük, biztosra mentünk, hogy az osztályok öröklődési jól voltak a sémában lefejlesztve, illetve az összes attribútumot is ellenőriztük.
Több kisebb figyelmetlenséget is észrevettünk, és javítottuk.
Több javítás után, újra próbáltunk tesztelni. Ebben az esetben egy másik, jobban beazonosítható hibát kaptunk futtatás során.
Már nem kaptunk Segmentation Fault hibát, hanem futás közben egy node-nál minden hibaüzenet nélkül leállt a program.
Kiderítettük, hogy melyik node-nál áll le a program mindig, ez volt maga a Literal node.
Ismét átnéztük a LiteralExpression osztályt, mivel ez maga a Literal, és egy rossz osztályból öröklődött le, ez okozta a program leállását.
Javítottuk a problémát, és ezek után hibamentesen és jól letudtuk futtatni.


Továbbá sok nyelvi elemet a TypeScript-ben nem ismertünk.
Tanulmányozni kellett jobban a TypeScript programozási nyelvet~\cite{fenton2014pro, cherny2019programming, nance2014typescript, 10.1007/978-3-662-44202-9_11}, hogy tesztek átnézése során tudjuk mit elemzett le jól és mit nem.

\chapter{Általam változtatott programok}\label{chap:altalam_valtoztatott_programok}

\section{JavaScriptAddon változások}

\noindent


A séma változtatás után, magát a NodeAddonGenerator fájlt nem módosítottam.
Ennek ellenére a JavaScriptAddon mérete lényegesen megnőtt, mivel a séma nagy bővítésen esett át.
A JavaScript-es nyelvi elemek mellett a TypeScript-es nyelvi elemek is megtalálhatóak benne.
Emiatt sokkal több node, és hozzájuk a wrapper-ek generálódtak le.
A JavaScript-es nyelvi elemek megmaradtak, néhány helyen még javítottunk hibákat is.

\section{JSAN2Lim Bővítések}

\noindent

A JSAN már TypeScript fájlokat és projekteket is tud elemezni, emiatt a JSAN2Lim-et is fejleszteni kellett.
A JSAN2Lim használja a JavaScript és a Lim ASG fájlokat egyaránt.

Több helyen is át kellett írni a JSAN2Lim-et:
\begin{itemize}
      \item Több visitor-nál a várt paraméter típusát átírni, például \texttt{Class}-ról \texttt{ClassDeclarationBase}-re,
      mivel a sémában már nem \texttt{Class}-ként, hanem \texttt{ClassDeclarationBase}-ként szerepel.
      \item Több visitor-nál a várt paraméternél helyét meg kellett változtatni egyes node-nál.
      Például a \texttt{RestElement} eddig a \texttt{Statement} package-en belül helyezkedett el, de a bővítés során átkerült a \texttt{Parameter} package-be.
      \item A \texttt{Pattern}, mint node, \texttt{Parameter}-re lett átnevezve. A helye is megváltozott, \texttt{Statement} package-ből átkerült a \texttt{Parameter} package-be.
      \item Típus lekérdezésnél a \texttt{getIsClass} helyett \texttt{getIsClassDeclarationBase} kellett használni.
\end{itemize}

A bővítések, hogy mikkel bővítettem a JSAN2Lim-et:
\begin{itemize}
      \item Több kind-ot is létrehoztunk a sémában, ezekkel bővíteni kellett a JSAN2Lim-et.
      \item Új node visitor-ok írása. Mint például a TSEnumDeclaration, TSImportEqualsDeclaration, TSInterFaceDeclaration, és még TypeScript node amihez kell visitor.
      A visitor-hoz a fillData függvényeket is megírni.
      \item JSAN2Lim-ben a kindStrings tömb kiegészítése TypeScript-es node-okkal.
      \item Hibák kijavítása, a VariableDeclaration-nél nem detektálta az összes változó deklarálást, főleg metódusokon belül.
      \item A \texttt{getLimKind} metódus bővítése \texttt{getIsTSTypeAliasDeclaration}, \texttt{getIsTSInterfaceDeclaration},
      \texttt{getIsTSAbstractMethodDefinition} és \texttt{getIsTSAbstractPropertyDefinition} esetekkel.
      \item TSEmptyBodyFunctionExpression osztály hibajavítása, hiszen több helyen is leállt a program emiatt.
\end{itemize}

\section{Ast binder optimalizálás}

\noindent

Az AST binder referencia kötésekre (továbbiakban: bindolásra) alkalmas. Ezek a referenciák a JSAN2Lim programhoz szükségesek.
Binder néven van definiálva a metódus a JSAN-ban.
Kétszer van használva, ezért is kulcsfontosságú, hogy gyors legyen.
Egyszer VariableUsages referenciákat bindol és egyszer ACG$^{~\cite{feldthaus2013efficient}}$  referenciákat bindol.

\noindent

A binder 4 argumentumot vár:

\begin{itemize}
      \item Egy stringet, ami lehet vagy VariableUsages (továbbiakban: VU) vagy ACG$^{~\cite{feldthaus2013efficient}}$.
      \item Absztrakt Szintaxis Fát (Abstract Syntax Tree, továbbiakban: AST), ami egyedien van felépítve.
      \item Egy tömböt, amiben JSON objektumok találhatóak, amik a linkeket tartalmazzák.
      \item Végül még egy stringet, ami lehet addCalls, vagy setRefersTo. Az AddCalls és a SetRefersTo a JavaScriptAddon-ban található meg és onnan hívódik meg.
      Abban az esetben kap addCalls értéket, ha ACG  referenciákat bindol a binder, és akkor setRefersTo, ha VU referenciákat.
\end{itemize}

A linkeket tartalmazó tömb egy JSON objektuma a következőképpen néz ki:

\begin{lstlisting}[caption={Binder JSON objektuma}, label={lst:binder_json_arg}, language={JavaScript}]
source: {
      label: IdentifierNeve,
      file: AbsPath,
      start: { row: <int>, column: <int>},
      end: { row: <int>, column: <int>},
      range: { start: <int>, end: <int>},
      node: [Object]
},
target: {...}
\end{lstlisting}

Ilyen JSON elemekből épül fel a tömb.
A forrás és a cél objektum felépítése ugyanaz, csak másak az értékek.
A kódrészletben a cél (source) van kifejtve bővebben.
A \texttt{label} az egy Identifier vagy PrivateIdentifier node nevét fogja jelölni.
A \texttt{file} egy abszolút útvonalat kap, ami az Identifiert tartalmazó fájlt jelöli.
A \texttt{start} az egy objektum, amiben külön van sor és oszlop meghatározva. Ez az Identifier első karakter pozíciójának a sor és oszlop értékei.
Az \texttt{end} ugyanaz, mint a \texttt{start}, csak itt az utolsó karakter pozíciójának a sor és oszlop értékei.
A \texttt{range} is egy objektum, a start ennél a kezdő karakter hanyadik karakter volt a kódban, és az end pedig az Identifier utolsó karakterének a karakterszáma.
A \texttt{node} is egy objektum ami maga az Identifier vagy PrivateIdentifier.

\subsection{Lassúság okai}

Binder-nek a futásideje lényegesen megnő, ha nagyon nagy projektekre futtatódik le.
Ennek több oka is van:

\begin{itemize}
      \item Minden node hívásnál meghívja a JavaScriptAddon-ból egy metódust.
      A JavaScriptAddon már magában nagy terjedelmű fájl. TypeScript támogatás után kétszeresére nőtt a mérete, mint ezelőtt, emiatt lassult is.
      \item Nagyobb projektekben több függvényhívás és változószám található az elemzett kódokban.
      \item Az AST nagyobb projekteknél nagyon nagy is lehet.
      Ezt az AST-t bejárjuk többször is bindolás alatt. Emiatt lesz nagyon lassú a binder.
      \item A JSON objektumokat tartalmazó tömb is nagy méretű lesz, de ez kevésbé lassítja a bindert, mint az AST bejárás.
\end{itemize}

\subsection{Optimalizálás}

\noindent

Optimalizálás során először a JavaScript Profiler-t használtam memória és futásidő vizsgálatra.
Nekem ez nem felelt meg, mivel a JavaScriptAddon-ban rengeteg minden egymásra épül, és 50-100 mélységű függvényhívásnál nem tudta kiírni a futásidejét és a memórihasználatot.
Ezután a binder metódust 3 részre bontottam \texttt{console.time} és \texttt{console.timeEnd} beépített függvénnyek segítségével.
Futásidőt írta ki a függvény a \texttt{console.time} és a \texttt{console.timeEnd} sorok között milliszekundumban.
Kiderült, hogy a következő sor felettébb nagy futási idővel rendelkezik:

\begin{lstlisting}[caption={Lassú metódus}, label={lst:binder_problemas_function}, language={JavaScript}]
getWrapperOfNode(resolveNode(astSet, sourceFile, element.source.range.start, element.source.range.end, true));
\end{lstlisting}

A resolveNode metódus első paraméterként vár egy AST-t, utána egy fájlnevet, kezdő- és végPozíciót, illetve egy igaz vagy hamis értéket.

Az AST-t forEach-el bejárja a program. Az AST egyes elemei az AST node-ok.
A forEach-en belül az AST node-on walk$^{~\cite{porter2006abstract}}$  metódus segítségével bejártuk, addig amíg nem találtuk meg a keresett node-ot.
Rosszabb esetben az AST legvégén volt a keresett node, mivel már a végén voltunk a bindolásnak.
Kisebb projektek esetén ez nem baj, mivel az AST mérete nem akkora, mint egy nagyobb projekt esetében.

\noindent

Optimalizálás során, létrehoztam indexAST metódust, ami indexeli az AST-t.

\begin{lstlisting}[caption={indexAST metódus}, label={lst:indexAST_function}, language={JavaScript}]
let indexAST = function (ast) {
      ast.forEach(astNode =>{
            globals.setActualFile(astNode.filename)
            walk(astNode, {
                  enter: function(node){
                  globals.setIndexed(astNode.filename, node.range[0], node.range[1], node)}})})
      return globals.indexedAST}
\end{lstlisting}

\Aref{lst:indexAST_function} kódrészletben látható, hogy egy AST-t várunk paraméterben. Ezt a bindertől fogja kapni.
A metódusban forEach-el bejárjuk az AST elemeit, amik az AST node-ok.
Beállítom a \texttt{globals.setActualFile()} függvénnyel az aktuális fájlnevet.
Ezután az adott AST node-ot walk függvény segítségével bejárjuk.
Csak az enter metódusát kellett szerkeszteni.
Ezután a setIndexed metódus meghívódik.
A setIndexed függvénynek megkapja a fájlnevet, a node range-nek a kezdő- és a végparaméterét, ahol kezdődik az adott node és hol végződik, karakterpontosan, és magát a node-ot.

\begin{lstlisting}[caption={setIndexed metódus}, label={lst:setIndexed_function}, language={JavaScript}]
const setIndexed = function(filename, range_start, range_end, node){
      let actualfilename = getFilePathAlt(filename)
      if (indexedAST[actualfilename + "-" + range_start + "-" + range_end] !== undefined && indexedAST[actualfilename + "-" + range_start + "-" + range_end] !== node){
            return
      }
      indexedAST[actualfilename + "-" + range_start + "-" + range_end] = node
}
\end{lstlisting}

A setIndexed először vizsgál arra, hogy az adott node be van-e már indexelve.
Ezt úgy teszi meg, hogy az indexedAST tömbben keres egy indexre.
Ez az index a következőképp néz ki: FájlNév-KezdőPozíció-VégPozíció.
Továbbá vizsgál arra is, hogy ha van ilyen indexű elem a tömbben, akkor ezen az indexen található-e már ilyen node.
Ha van akkor nem állít be semmit, csak kilép, mivel már be van indexelve az adott node. Ha nincs, akkor beállítja az indexedAST tömbnek az adott indexre az adott node-ot.
Az AST-t csak egyszer járjuk be a foreach-el. A bejárás után minden node be lesz indexelve a tömbbe.

\noindent

\Aref{lst:binder_problemas_function} kódrészleten látható, hogy először a resolveNode függvény segítségével megkerestük a keresett node-ot kezdő- és végPozíció alapján.
Ezután a megkapott node-ra meghívtuk a \texttt{getWrapperOfNode} függvényt, hogy megkapjuk a wrapperjét a JavaScriptAddon-ból.
A keresést megváltoztattam, írtam rá egy getIndexed metódust.

\noindent

A \texttt{getIndexed} függvény megvizsgálja, hogy forrás vagy cél node-ot keresünk.
Ha cél node-ot, akkor visszaadjuk a \texttt{getWrapperOfNode(indexedAST[FájlNév-KezdőPozíció-VégPozíció])}-t.
Ha cél node-ot keresünk, akkor a walk függvény segítségével bejárjuk a beindexelt tömböt.
Ez lényegesen gyorsabb, mint a resolveNode-os megoldás, mert ott minden egyes esetben az egész AST-t bejártuk a walk függvény segítségével, jelen esetben csak a beindexelt tömb elemét járjuk be.
Ha megkaptuk a cél node-ot akkor visszaadjuk azt a \texttt{getWrapperOfNode(result)} hívással.

\subsection{Eredmények}

\begin{table}[h!]
      \begin{center}
      \caption{Eredmények összehasonlítása}\label{tbl:Eredmenyek}
      \begin{tabular}{c c}
      \hline Optimalizálás előtt (m:ss.mmm) & Optimalizálás után (m:ss.mmm)\\
      \hline 42:31.211 & 1:39.140\\
      \hline 2:51.045& 0:29.922\\
      \hline 1:17.825& 0:3.949\\
      \hline 0:32.5& 0:0.614\\
      \hline 0:8.2 & 0:0.120\\
      \hline 0:4.830& 0:1.423\\
      \hline 0:4.8 & 0:0.153\\
      \hline 0:0.4 & 0:0.2\\
      \hline \\
      \end{tabular}
      \end{center}
\end{table}

Egy nagyobb projektre futtattam le a JSAN-t, 2 közepesre és 5 kisebbre. Az eredmények láthatóak \aref{tbl:Eredmenyek} táblázatban.
Eredménynek ugyanazt adta mind a kettő futás, viszont futásidőben nagyon eltértek egymástól.

\chapter{Regressziós tesztek frissítése}\label{chap:Regteszt_frissítés}

\section{Tesztek kibővítése}
Először kezdtem a JSAN tesztek átírásával. A JSAN-nak a következő kapcsolói voltak:
\begin{itemize}
      \item -i: Jelentése input, lehet relatív vagy abszolút útvonal a fáljoz vagy projekthez.
      \item -o: Jelentése output neve, lehet relatív vagy abszolút útvonal.
      \item -d: Jelentése dumpjsml, a JSAN eredményét átgenerálja XML stílusú fájlba és ezt egy jsml fájlba kiírja.
      \item -e: Jelentése ExternalHardFilter, relatív vagy abszolút útvonal egy olyan fájlhoz, ami szövegalapú és olyan szintaxis található benne, ami kell az externalHardFilternek.
      \item -help: Kiírja minden kapcsolóhoz tartozó leírást.
      \item -r: Jelentése useRelativePath, eredményben az útvonalakat átírja relatív útvonalra.
      \item -h: Jelentése html, a JSAN html fájlokra is lefut, bennük keresve JavaScript-es szkripteket és azokat teszteli.
      \item -stat: Jelentése statistics, kiírja a memóriahasználatot és a futásidőt amit a JSAN vett igénybe.
\end{itemize}


Az input, output, dumpjsml, ExternalHardFilter, és html kapcsolókra tudtam tesztelést írni. A logika az volt, hogy mappanév alapján tesztelek egyes kapcsolókra.
ProgramozásiNyelv-kapcsolónév logikát követtem, mivel JavaScript-es tesztek mellé kell majd TypeScript-es teszteket is keresnem.
A projekteket python szkriptek segítségével futtattam le.

\begin{lstlisting}[caption={JSAN kapcsoló vizsgálat pythonban}, label={lst:python_kapcsolo}, language={Python}]
if "externalHardFilter" in input_path:
      ret_val = self._execute_one_test(input_path, external_hard_filter=True) and ret_val
      return ret_val
\end{lstlisting}

\Aref{lst:python_kapcsolo}-es kódrészleten látható, hogy hogyan keresek egy adott kapcsolóra. Az input\_path-ban van a mappa is, és a js-externalHardFilter-ben megtalálható az externalHardFilter szó.
Az execute\_one\_test függvényemben beállítom a teszteléshez az adott dolgokat.

\begin{lstlisting}[caption={JSAN kapcsoló beállítása pythonban}, label={lst:python_kapcsolo_beallitasa}, language={Python}]
if external_hard_filter:
      input_dir = os.path.dirname(input_path)
      external_hard_filter_path = os.path.join(input_dir, "externalHardFilter.txt")
      external_hard_filter_switch = "-e"
else:
      external_hard_filter_path = ""
      external_hard_filter_switch = ""
\end{lstlisting}

\Aref{lst:python_kapcsolo_beallitasa}-es kódrészletben az execute\_one\_test függvénynek egy részét láthatjuk, ahol beállítjuk az external\_hard\_filter\_path-t és a kapcsolót annak függvényében, hogy igazat kaptunk e vagy sem.
Létrehozzuk az externalHardFilter fájlt a tesztelendő fájl mellé vagy a tesztelendő projekt gyökerébe,
és attól függően, hogy ki akarjuk hagyni az adott fájlt vagy hozzáadni, írunk egy $+$ vagy egy $-$ jelet a sor elejére és utána relatív útvonal és a fájl neve.
Alapértelmezetten minden fájl elemez az adott program, JSAN esetében ezek azok a fájlok amiknek a kiterjesztése \texttt{js}, \texttt{jsx}. (külön kapcsolóval megadhatjuk neki, hogy a html kiterjesztésű fájlokat elemezze-e vagy se.)
TypeScript-es bővítés után már a ts és a tsx kiterjesztésű fájlokat is elemzi. Ezért írtam több tesztesetet is. Egy példa, hogy hogyan néz ki ez a fájl:

\begin{lstlisting}[caption={ExternalHardFilter fájl}, label={lst:external_hard_filter}]
-filtered01
-filtered02
-filtered03
+filtered01
\end{lstlisting}

\Aref{lst:external_hard_filter}-as kódrészletben látható, hogy filtered01, filtered02 és filtered03 at kivettük, hogy azokat ne elemezze a jsan.
Ezután visszavettük a filtered01-et. A filtered fájlok azok JavaScript-es fájlok, JavaScript-es kóddal.
Ezután referenciába csak a filtered01 fájl eredményét raktam be, hiszen a 02 és 03-at nem elemzi, ha elemezné, akkor szólna a program, hogy hiányzó referecia fájl.
Természetesen lehet regurális kifejezést is használni filterezésnél.


Ezután teszteltem a \texttt{-i} kapcsolót, itt a programnak vissza kellene adnia, hogy üres az input ha nincs megadva.
Ezután a \texttt{-o} kapcsolóra teszteltem, itt ebben az esetben alapértelmezett értékben \texttt{out.jssi} fájlt kellene visszaadnia.
Végül a \texttt{-h} kapcsolót néztem meg, itt ha megvan adva ez a kapcsoló, akkor az adott projektben a html fájlokban a JavaScript-et kellene tesztelni.
A \texttt{-d}, \texttt{-help}, \texttt{-stat} kapcsolókra nem tudtam tesztelni, még a \texttt{-useRelativePath} kapcsolóra sem, hiszen ha abszolút utat kérek, akkor a referenciákban másnál rossz lesz az elvárt eredmény.


Ezután a JSAN2Lim és az ESLintrunner programoknak írtam át a tesztelési menetetét, mivel nekik volt még olyan kapcsoló, amit lehetett tesztelni. A többi projektnek 1 vagy 2 volt, amik kellettek a működésükhöz, nem voltak opcionális kapcsolók.

Miután a JSAN ki lett egészítve TypeScript támogatással és a JSAN2Limet átírtam, elkezdtem teszteket keresni, mind a JSAN-nak és mind JSAN2Lim-nek.
Kerestem egyszerűbb TypeScript és picit összetettebb TypeScript projekteket is, hogy lássam az esetleges hiányosságokat.
Természetesen ami eredményeket adtak a programok, azokat át kellett néznem egyesével, hiszen csak így tudom meg, hogy jól tesztelte-e a megadott fájlt vagy sem.
Több kisebb hiányosságot is észrevettem, mind JSAN oldalról, mind JSAN2Lim oldalról, ezek kisebb hiányosságok voltak.
Például, hogy egy node-nak nem volt beállítva pozíció, vagy nem volt jól beállítva a paramétere.

\chapter{Összefoglaló}\label{chap:Összefoglaló}

\section{Program javulása}

\noindent

A szakdolgozatom során sikerült elérnem azt, hogy az Ast binder közel tízszer gyorsabban fusson le nagyobb projektre, mint ezelőtt.
Leteszteltem ugyanarra a nagy projektre az eredeti JSAN lefutását és az én általam átírt lefutását. Ezt \Aref{lst:jsan_before_after_comparison} kódrészleten láthatjuk.
Kisebb projektekre körülbelül ötszörös gyorsulás van.
Lefuttattam egy nagyobb projektre a jsan-t, az optimalizált és az optimalizálatlan verzióval, hogy jobban lássuk a különbséget. A következő eredmények születtek:

\begin{lstlisting}[caption={JSAN lefutási idő előtte és utána}, label={lst:jsan_before_after_comparison}]
Optimalizalt_verzio:
Binding VU: 71599/71599
VU binding: 1:39.140 (m:ss.mmm)

Optimalizalatlan_verzio:
Binding VU: 71599/71599
VU binding: 42:31.211 (m:ss.mmm)
\end{lstlisting}

Ugyanazt az outputot adja mind a kettő program, szóval csak gyorsaságban és memóriahasználatban változott sokat.

Emellett a JavaScriptSchema újraírása is sikeres volt, emiatt a JSAN tud typescriptes kódokat elemezni.
JSAN2Limet sikeresen módosítottam, hogy a JSAN által kiadott outputot sikeresen limmé alakítsa.
\section{Jövőbeli tervek}

\noindent

JavaScriptSchema bővítése sok időbe telt, mivel se dokumentáció, se tapasztalat nem volt. Ezután még a JSAN2Lim átírás is sok időbe került.
Eközben a typescript-eslint github repo amit használtunk a bővítésre, frissült, sok új funkciót hoztak be.
Rengeteg változtatás volt a typescript résznél, de a schémát nem tudtuk még naprakészre hozni, mivel voltak ennél fontosabb feladatok.
Egyik jövőbeli terv az, hogy a schémát up to datere hozzam, mivel már van hozzá dokumentáció is, meg nagyjából én is írtam, ezért nem lesz ez annyi idő, mint volt az elején az újraírása.

\noindent

Végül a JSAN programra még bőven ráfér az optimalizálás, mivel csak az ast bindert optimalizáltuk, rengeteg helyen feleslegesen van bejárva az ast.
Ez a másik jövőbeli terv.

\include{tex/Nyilatkozat}
\addcontentsline{toc}{chapter}{Irodalomjegyzék}
\bibliographystyle{plain}
\bibliography{references}

\end{document}
