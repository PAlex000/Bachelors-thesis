\documentclass[12pt,a4paper]{report}
\usepackage[margin=2.5cm]{geometry}
\usepackage[magyar]{babel}

% magyar nyelv tamogatas
\usepackage{t1enc}
\usepackage[T1]{fontenc}
\usepackage[utf8]{inputenc}

% A formai kovetelmenyekben megkövetelt Times betűtípus hasznalata:
\usepackage{times}

\usepackage{setspace}
\usepackage{listings,multicol}
\usepackage{xcolor}
\usepackage{textcomp}
\usepackage{enumitem}
\usepackage{booktabs}
\usepackage[unicode,hidelinks]{hyperref}
\usepackage{footnote}
\usepackage{ifthen}

% Törölhető package
\usepackage{lipsum}

% TODO csomag, amivel jól észrevehető todokat hagyhatunk a dolgozatban
\usepackage{todonotes}
% Inline TODO
\newcommand{\todoi}[1]{\todo[inline]{\textbf{TODO:} #1}}

% egyedi lablec
\usepackage{fancyhdr}
\usepackage{graphicx}
\graphicspath{{fig/}}

% Kódrészletes színezése
\definecolor{lightgray}{rgb}{.9,.9,.9}
\definecolor{darkgray}{rgb}{.4,.4,.4}
\definecolor{purple}{rgb}{0.65, 0.12, 0.82}

\lstdefinelanguage{JavaScript}{
  keywords={typeof, new, true, false, catch, function, return, null, catch, switch, var, if, in, while, do, else, case, break},
  keywordstyle=\color{blue}\bfseries,
  ndkeywords={class, export, boolean, throw, implements, import, this},
  ndkeywordstyle=\color{darkgray}\bfseries,
  identifierstyle=\color{black},
  sensitive=false,
  comment=[l]{//},
  morecomment=[s]{/*}{*/},
  commentstyle=\color{purple}\ttfamily,
  stringstyle=\color{red}\ttfamily,
  morestring=[b]',
  morestring=[b]"
}

\lstset{
   language=JavaScript,
   backgroundcolor=\color{lightgray},
   extendedchars=true,
   basicstyle=\footnotesize\ttfamily,
   showstringspaces=false,
   showspaces=false,
   numbers=left,
   numberstyle=\footnotesize,
   numbersep=9pt,
   tabsize=2,
   breaklines=true,
   showtabs=false,
   captionpos=b
}

\lstdefinelanguage{CStyle}{
    backgroundcolor=\color{lightgray},   
    commentstyle=\color{purple}\ttfamily,
    keywordstyle=\color{blue}\bfseries,
    numberstyle=\footnotesize,
    stringstyle=\color{red}\ttfamily,
    basicstyle=\footnotesize\ttfamily,
    breakatwhitespace=false,         
    breaklines=true,                 
    captionpos=b,                    
    keepspaces=true,                 
    numbers=left,                    
    numbersep=9pt,                  
    showspaces=false,                
    showstringspaces=false,
    showtabs=false,                  
    tabsize=2,
    language=C
}

\renewcommand{\lstlistingname}{Kódrészlet}

% Margók beállítása
\hoffset -1in
\voffset -1in
\oddsidemargin 35mm
\textwidth 150mm
\topmargin 15mm
\headheight 10mm
\headsep 5mm
\textheight 237mm

% Szerző és dolgozat adatai
%Szerző adatai
\newcommand{\nev}{Pozsgai Alex}
\newcommand{\szak}{programtervező informatikus BSc}
\newcommand{\tanszek}{Szoftverfejlesztés}
\newcommand{\ev}{2023}
\newcommand{\dolgozatTipusa}{Szakdolgozat}
\newcommand{\vegsoDatum}{\today}

\newcommand{\cim}{Ipari JavaScript elemző kiegészítése TypeScript támogatással}
\newcommand{\angolcim}{Industrial JavaScript analyzer enhancement with TypeScript support}

%Témavezető adatai
\newcommand{\temavezetoNev}{Dr.Antal Gábor}
\newcommand{\temavezetoBeosztas}{Tudományos munkatárs}


\begin{document}

% Másfeles sorköz
\setstretch{1.5}
\sloppy

\pagestyle{fancy}
\fancyhf{}
\fancyhead[L]{\textit{\cim}}
\fancyfoot[R]{\thepage}
\fancypagestyle{plain}{%
    \renewcommand{\headrulewidth}{0pt}%
    \fancyhf{}%
    \fancyfoot[R]{\thepage}%
}

\include{tex/0_Title}

\pagenumbering{arabic}
\chapter*{Feladatkiírás}
\addcontentsline{toc}{section}{Feladatkiírás}

\noindent

A jelenlegi projektet SourceMeter JavaScript-nek hívják.
A cél az, hogy ez a projekt többet tudjon, mint a piacon a többi kódelemező.
Ebből adódóan bővíteni kell TypeScript támogatással.

\noindent

A hallgató feladata ennek a programnak a fejlesztése úgy, hogy tudjon TypeScript forráskódot, fájlokat és projekteket is kellő pontossággal elemezni.
Ezután az eszközcsalád optimalizálása.

\chapter*{Tartalmi összefoglaló}
\addcontentsline{toc}{section}{Tartalmi összefoglaló}

\noindent\textbf{A téma megnevezése:}

JavaScript Analyzer kiegészítése TypeScript supporttal, JSAN optimalizálása.

\noindent\textbf{A megadott feladat megfogalmazása:}

A feladat során el kell érni azt, hogy a JavaScript Analyzer Tool tudjon TypeScript fileokra lefutni, és azokat nagy pontossággal elemezni. 
Emellett a JSAN működését optimalizálni.

\noindent\textbf{A megoldási mód:}

A megoldás során kell változtatni a JavaScriptSchémán, és a JSAN-ban az AstTransformer.js fileban.

\noindent\textbf{Alkalmazott eszközök, módszerek:}

A megoldáshoz Visual Studio Code IDE-t, a JavaScriptSchema szerkesztéséhez Visual paradigm-t használta.
A projekt lebuildeléséhez és ellenőrzéséhez Visual Studio 2017 programot használtam.

\noindent\textbf{Elért eredmények:}

JSAN képes TypeScript fileokat nagy pontossággal elemezni, nagyobb projektre lényegesen gyorsabban fut le, kevesebb erőforrást igényel, mint előtte.

\noindent\textbf{Kulcsszavak:}

JavaScriptAnalyzer, TypeScriptAnalyzer, JavaScript, TypeScript, Optimalizálás, Visual Studio Code, Visual Paradigm, Vpp, C++

\include{tex/0_TartalomJegyzek}


\chapter*{Bevezetés}
\addcontentsline{toc}{chapter}{Bevezetés}

\noindent

Szakdolgozatomban a JavaScript Analyzer továbbfejlesztése a cél.
A továbbfejlesztés arról szól, hogy a javascript mellett előtérbe jön a typescript is.
Ezáltal a typescript fileokat és projekteket is elemezni kell.
Ez azt eredményezi, hogy a Javascript Analyzert gyökerestül át kell írni, annak érdekében,
hogy JavaScriptet és TypeScriptet is tudjon egyaránt elemezni.

\noindent

A JavaScript Analyzer az egy nagyobb projektnek az egyik alprojektje. Ezt a nagyobb projektet Analyzer-JavaScriptnek hívják.
Az Analyzer-Javascriptnek több alprojektje is van, amiben sok minden egymásra épül.
A szakdolgozatomban a következő alprojektekben fogok változtatni és bővíteni: JavaScriptSchema, JSAN (ezen belül is csak az astTransformerben), és az erre épülő JSAN2Lim.
Emellett még megtalálhatóak a következő alprojektek is: ESLintRunner, ESLint2Graph, LIM2Metrics, LIM2Patterns, DuplicatedCodeFinder és a ChangeTracker.
Azért kell a JSAN2Limben is változtatni, mivel ha gyökerestül megváltoztatom a JSAN-t,
akkor a JSAN2Lim rossz eredményeket fog visszaadni a typescript fileok elemzése közben.

\noindent

A projekten amin fejlesztettem, többen is fejlesztettek egyszerre, ezáltal voltak átfedések, oszthatatlan részek.
A projekt néhány része egymásra épül, ezért kiemelten fontos volt a csapatmunka egyes részeknél, hogy a projekt eredményesen és hatékonyan haladjon.
Az eredményes csapatmunka magában foglalja az eredmények és az előrehaladás folyamatos kommunikálását,
ami azt jelenti, hogy mindketten megértjük, hogy milyen célkitűzések vannak a projektben,
és rendszeresen jelentést adunk egymásnak az elvégzett munkáról és az elért eredményekről.

\noindent

Fejlesztés során JavaScriptSchema file szerkesztése csoportos munka volt, hiszen minden erre épült, ez volt az alapja mindennek.
Sok időt vett igénybe a szerkesztés. Nulláról kellett újraírni az egész vpp filet.
Sajnos dokumentáció nem készült az előző filehoz, ezért nekünk kellett kitalálni, hogy mi mit csinált, hiszen akik ezt írták, ők már nem foglalkoztak ezzel és elérhetetlenek voltak.
A szerkesztéshez szükséges volt a Visual Paradigm alkalmazást használni.
Ezzel egyikőnk sem találkozott még, szóval először ezt kellett tanulmányozni, megérteni.
Ezután értelmezni kellett a meglévő schemát, hiszen eddig az jól működött, csak nem lehetett könnyen bővíteni.
Mérlegeltük a két opciót, ahol vagy megpróbáljuk bővíteni a jelenlegi schémát, vagy nulláról elkezdjük újraírni.
Végül az újraírás mellett döntöttünk, hiszen ezt láttuk gyorsabb és könnyebb megoldásnak. Egy könnyen bővíthető, dokumentál schema volt az elképzelés.
Ketten fejlesztettük le végül ezt a schemát.

\noindent

Ezután változtattam a JSAN2Lim-et, hogy képes legyen értelmezni a JSAN outputjait a typescriptes elemzések során.
Itt át kellett néznem a c++ programozási nyelvet, hiszen az egész program ebben íródott, ugyanúgy dokumentáció nélkül.
Több nehézségbe is ütköztem, de a legvégén sikerült olyan állapotra hozni a programot, hogy az output nagy százalékát sikeresen átkonvertálja lim fileba.

\noindent

Ezután a JSANnak az astTransformer filejában a bindert optimalizáltam.
Értelmezni kellett a kódot, át kellett néznem az ast-t, meg több adatszerkezetet is, gyorsasági szempontból.

\noindent

Legvégül pedig miután minden készen volt, bővítettem a regteszteket új projektek behozatalával, mind javascripteseket és typescripteseket.
Ezeknek az outputjait megnéztem, hogy jó eredményt ad-e, itt még volt egy kis bugfixing a JSAN2Limben és a Schémában.

\chapter*{JavaScriptSchema}\stepcounter{chapter}
\addcontentsline{toc}{chapter}{JavaScriptSchema}
%\chapter{JavaScriptSchema átírása}\label{chap:JavaScriptSchema átírása}
\section{A JavaScriptSchemáról}

\noindent

A JavaScriptSchéma egy UML Diagramhoz hasonlító schéma. Jelen esetben a Visual Paradigm alkalmazással szerkeszthető.
A felépítése a következőképpen alakul:

\begin{figure}[!htbp]
      \caption{JavaScriptSchema struktúrális felépítése}\label{fig:JavaScriptSchema_struktura}
      \centering
      \includegraphics[width=0.4\textwidth]{JavaScriptSchema_struktura.png}
\end{figure}

\Aref{fig:JavaScriptSchema_struktura} ábrán a következő struktúra figyelhető meg: ProjektNév-Model-Packagek.
A projekt neve a JavaScript, ezen belül található egy model, amit javascript-nek hívnak. A modellen belül találhatóak meg a packagek.
A Packagek nem véletlenül így lettek elnevezve. Az alábbi hivatkozáson található meg, hogy milyen logika alapján neveztük el a packageket:~\cite{typescript-eslint}
Nem feltétlenül muszáj több packaget létrehozni, ez csupán az átláthatóság és a könnyen bővíthetőség céljából lett így megvalósítva.
Követtük a typescript-eslint githubon lévő projekt struktúrális logikáját, több helyen is eltértünk tőle, mivel a mi projektünk másabb, illetve speciális megoldásokat igényelt egyes helyeken.
A következő oldalakon bemutatom a packagek felépítését néhány egyszerűbb példán szemléltetve.
A packagek a következőképpen épülnek fel:
\begin{figure}[!htbp]
      \caption{A base package felépítése}\label{fig:base_vpp}
      \centering
      \includegraphics[width=0.9\textwidth]{base_vpp.png}
\end{figure}

\Aref{fig:base_vpp} ábrán különböző osztályok láthatóak. A Base osztály mindennek az alapja, ebből öröklődik minden más.
Látható, hogy egy id attribútummal rendelkezik, aminek a típusa NodeId.
A zöld háttérszínű osztályok az absztrakciót jelentik. Működésben nincs jelentősége, csak a schéma szerkeszthetősége és olvashatósága miatt van így jelezve.
A szürke háttérszínű osztályok pedig csupán annyit jelentenek, hogy más packageben lettek definiálva.
A kék a default, normális osztályt jelölik.
A Positioned és a System származik le a Baseből, értelemszerűen a system az maga a program lesz.
A Positioned az azért absztrakt, mivel majd ebből fognak leszármazni a kisebb osztályok, mint például az expression, statement és a többi.
A Positioned osztálynak van egy attribútuma, a poisition, ami egy Range típus. Emellett még tartozhatnak hozzá kommentek is.
A komment egyaránt leszármazik a positioned-ből, ezzel garantálva azt, hogy a kommentnek van pozíciója.
Látható, hogy a kommentnek van egy text, type és location attribútuma. A CommentType és a CommentLocation a DataStructures-ben van definiálva.
Emellett a Named, BaseNode és a BaseToken származik le a Positionedből. A Named osztály azt jelenti, hogy egy nodenak van-e name attribútuma vagy sem.
A BaseNodeból és a BaseTokenből fog nagyon sok minden leszármazni aminek van typeja.
Végül, megtalálható a Program osztály, aminek van name attribútuma, mivel Named-ből származik le, és van Systeme. Az 1..* jelenti azt, hogy legalább 1 Systeme van, de lehet több is.
A hasPrograms-nak majd máshol lesz jelentősége, a mi esetünkben majd a javascriptben.
Tetszőlegesen el lehet nevezni, de konzisztencia miatt, minden attribútum ami osztály (és nem a DataStructuresben azon belül is a Kindsban van definiálva, hanem van neki egy osztály, mint pl Comment)
azt a hasOsztály névvel fogjuk ellátni.
A ProgramSourceType is a Kinds packageben van definiálva, ami csupán azt mondja meg, hogy az adott program vagy script vagy module típusú.
A base package nem követi a typescript-eslint githubon lévő projekt base mappáját, ez teljesen egyedi, átememeltük az egészet apróbb módosítással az előző projekt verzióból.

\noindent

A base packagen kívül bemutatom a declaration packaget, hogy lássuk milyen logikát követtünk a megírás során.
Ezen belül is az ExportAllDeclaration bemutatása:

\begin{lstlisting}[caption={ExportAllDeclaration typescriptes megvalósítása},label={lst:ExportAllDeclaration}, language={JavaScript}]
export interface ExportAllDeclaration extends BaseNode {
      type: AST_NODE_TYPES.ExportAllDeclaration;
      assertions: ImportAttribute[];
      exported: Identifier | null;
      exportKind: ExportKind;
      source: StringLiteral;
}
\end{lstlisting}

\Aref{lst:ExportAllDeclaration} kódrészlet így lett megvalósítva a JavaScriptSchémában:

\begin{figure}[!htbp]
      \caption{A declaration package felépítése}\label{fig:declaration_vpp}
      \centering
      \includegraphics[width=1\textwidth]{declaration.png}
\end{figure}

\Aref{fig:declaration_vpp} ábrán látható az, hogy minden a BaseNodeból származik. \Aref{lst:ExportAllDeclaration} kódrészletben ez az extends BaseNode.
Itt maga a declaration osztály az a DeclarationStatement névre hallgat, csupán azért, mert követtük a typescriptes kódot. A főbb osztályok a BaseNode alatt találhatóak.
Jobb oldalt a sötétszürke és a világosszürke osztályok azok az attribútumok.
Ebben az esetben az ExportAllDeclaration osztály leimplementálását mutatom be a JavaScriptSchemában.
Az ExportAllDeclaration öröklődik a Statement, DeclarationStatement, Node és a ProgramStatementből.
Ezeket az öröklődéseket ugyanúgy a typescript-eslint githubról néztük, unions mappában érhetőek el.
Ezáltal megkapja az összes szülőnek a tulajdonságait. A DeclarationStatement öröklődik a BaseNodeból, ami azt jelenti, hogy a BaseNode tulajdonságait is megkapja az ExportAllDeclaration.
A BaseNode ugye meg származik a Positionedből, ez látható \aref{fig:base_vpp} ábrán.
Emiatt az ExportAllDeclaration-nek lesz pozíciója, kommentje és NodeId-ja is.
Assertions attribútum típusa az ImportAttribute, látható, hogy egy tömböt vár, ezért 1..* a multiplicityje.
Exported attribútumnál egy Identifier típusú attribútumot vár, itt mi kiegészítettük még egy LiteralExpressiönnel is, ami maga a Literal.
A vagyolást egy OR-al jeleztük a JavaScriptSchemában.
Mivel lehet null-is, ezért 0..1 a multiplicity, szóval vagy 0 vagy 1.
Az ExportKind az a Kindsban található meg, így szimplán csak arra hivatkozunk, mint ExportAndImportKind.
Végül a source attribútuma egy StringLiteral, nálunk is így szerepel.
Természetesen ami sötétszürkével van jelölve, az máshol létre van hozva és vannak neki attribútumai.
A világosszürke annyit jelöl, hogy ebben a packageben lett létrehozva az osztály, de láthatóság szempontból többször szerepel, mivel lehet attribútum is.
Így lett minden egyes osztály felépítve, természetesen \Aref{fig:declaration_vpp} ábra csak egy részlete a declaration packagenek, ennél jóval nagyobb.
Vannak úgynevezett gyűjtő osztályok is, mint pl a Statement, ezekre azért van szükség, mert majd később ha le lett generálva minden, akkor tudjunk majd vizsgálni különböző nodeokra, mint például az isStatement.

\noindent

Sokat említettem a DataStructurest. Hadd mutassam be egy példán keresztül, hogy ez hogy néz ki:

\begin{figure}[!htbp]
      \caption{A DataStructures packageben a Kind package felépítése}\label{fig:data_structures_kinds}
      \centering
      \includegraphics[width=0.6\textwidth]{data_structures.png}
\end{figure}

A DataStructures packageben található 2 package, ez \aref{fig:JavaScriptSchema_struktura} ábrán látható. Ebből a Kinds packaget mutatom be.
\Aref{fig:data_structures_kinds} ábrán látható egy kis szelet a Kinds packageből.
Enumok találhatóak ebben a packageben.
Például ha az ExportAndImportKind-ot adjuk meg típusnak az egyik attribútumnak, akkor az lehet vagy eaiType vagy eaiValue.
Minden constant előtt van 3 karakter, ez azért szükséges, mert később ezekkel még foglalkozni fogunk.

\noindent

A JavaScriptSchema magában még semmit sem csinál. Először ki kell exportálni az egész projectet xml formátumba.
Ezután az xml filet átkonvertáljuk egy asg filera. Ez úgy történik, hogy a JavaScript Analyzer projektnek van egy kisebb alprojektje, amit UmlToAsg-nek hívnak.
Ez a java kód átkonvertálja az xml fileban lévő adatot egy asg fileba. Jobban nem térnék ki az UmlToAsg projektre, mivel nem szerkesztettük.
Az UmlToAsg projektet a SchemaGenerator generálta le. A SchemaGenerator c++ nyelven megírt program. Több mindent is generál c, c++, vagy java nyelven.
Ez a projekt is a JavaScript Analyzer alprojektei közé tartozik.
A legenerált asg file a következőképp néz ki:
A file elején a következő található meg:
\begin{lstlisting}[caption={Asg file első sorai},label={lst:asg_file_eleje}, language={JavaScript}]
NAME = javascript;

APIVERSION = 0.3.1;
BINARYVERSION = 0.3.1;
CSIHEADERTEXT = JavaScriptLanguage;
\end{lstlisting}

A verziókat kézzel tudjuk átírni abban a fileban ami generálja ezt, ez egy c++ file, a SchemaGeneratorban található meg.
Ezután a Kinds mappa tartalmát írja bele a következőképpen:
\begin{lstlisting}[caption={Asg file kind},label={lst:asg_file_kinds}, language={JavaScript}]
KIND ASTNodeTypes (ant) {
      FunctionDeclaration;
      BlockStatement;
      ClassDeclaration;
      TSTypeParameterDeclaration;
      TSTypeParameter;
}
\end{lstlisting}
Az a 3 karakter amit minden constant elé tettünk azt kitette paraméterbe és csak az utáni stringet írta át. Ugyanígy van a többi kindnál is.
Ha az összes kindot beleírta, akkor kezdi írni sorban a többi packaget. \Aref{fig:JavaScriptSchema_struktura} alapján megy sorba.
Példának a declaration packaget mutatom be, azon belül is az ExportAllDeclaration-t.
\begin{lstlisting}[caption={Asg file ExportAllDeclaration},label={lst:asg_file_export_all_declaration}, language={JavaScript}]
SCOPE declaration {

      NODE DeclarationStatement : virtual base::BaseNode [ABSTRACT] {
      }

      NODE ExportAllDeclaration : DeclarationStatement, statement::Statement, virtual statement::ProgramStatement, special::Node {
            ATTR ExportAndImportKind exportKind;
            EDGE TREE 1 hasExported (expression::Identifier | expression::LiteralExpression);
            EDGE TREE 1 hasSource (structure::StringLiteral);
            EDGE TREE * hasAssertions (special::ImportAttribute);
      }
}
\end{lstlisting}
Látható, hogy a Packaget SCOPE-nak értelmezi, és ezen belül NODE-ok találhatóak.
A Nodeoknak ATTR és EDGE TREE van. A vagyolás is látható.
Az öröklődik egy kettőspont után, felsorolás szerűen írta át, ha valami másból származik, akkor packageNev::osztalyNev szerint.
Az kapott ATTR jelölést ami a Kinds-ban megtalálható vagy egy szimpla típus (mint pl string, int).
Minden mást EDGE TREE-nek nevezett el. Az Edge tree utáni szám vagy csillag az a multiplicitást jelenti.
Ha 1es, akkor a multiplicitás 0..1, ha *, akkor vagy 0..* vagy 1..* a multiplicitás.
\Aref{fig:declaration_vpp}as ábrán látható, hogy mit hogyan írt át.

\section{JavaScriptSchema átírása}

\noindent

A JavaScript Analyzer a JavaScriptSchemára épül. Ha valamit meg akarunk változtatni gyökerestül a JavaScript Analyzerbe, akkor a schémát is változtatni kell.
Azt elérni, hogy javascript mellett még typescriptes kódokat is elemezzen a JavaScript Analyzer, ahhoz gyökerestül meg kellett változtatni a JavaScript Analyzert.
Az előző JavaScriptSchema (ami csak javascriptet elemzett) az a javascript hivatalos oldala alapján készült. Ami ábrák fentebb megtalálhatóak, azok már az átírt schémából vannak.
Több opció is volt, hogy most vagy legyen átírva a schéma, vagy legyen egy külön typescriptre is írva.
Végül rájöttünk, hogy ha van egy typescriptes schémánk, az képes javascriptet ugyanúgy elemezni ha megfelelően van lefejlesztve.
Így egy schémát fejlesztettünk le. Az előző schéma átláthatatlan volt, ezért úgy döntöttünk, hogy majdnem a nulláról újraírjuk.
Egyedül a base packaget emeltük át a régiből, BaseNode-al és BaseTokennel kibővítve (\Aref{fig:base_vpp} ábrán látható).
Emellett el kellett dönteni, hogy milyen struktúrát kövessünk, ami jól átlatható és később könnyebben bővíthető.
Végül a typescript-eslint official github alapján haladtunk. Annyi változtatással, hogy ami nekik a base mappában volt, mi arra létrehoztunk egy külön structure packaget.
Mindenhez készítettünk dokumentációt, jól érthetően leírtuk, hogy mit miért kellett csinálni.
Mivel minden is egymásra épül a typescriptes schémában, ezért nem tudtuk tesztelni minden egyes package után, hogy működik-e vagy sem.
Mint például, látható \aref{lst:asg_file_export_all_declaration} kódrészleten, hogy az ExportAllDeclaration mennyi mindenből származik le vagy mennyire sok attribútuma van.
Az ExportAllDeclaration egyik attribútuma, az assertions típusa az ImportAttribute. Ahhoz, hogy jól észrevegye az ExportAllDeclarationt az Analyzer, ahhoz jól le kellett fejleszteni az ImportAttributet.

\begin{lstlisting}[caption={ImportAttribute},label={lst:asg_file_import_attribute}, language={JavaScript}]
export interface ImportAttribute extends BaseNode {
      type: AST_NODE_TYPES.ImportAttribute;
      key: Identifier | Literal;
      value: Literal;
}
\end{lstlisting}
Ahhoz, hogy az ImportAttributet letudjuk fejleszteni a schémában, ahhoz le kell fejleszteni az Identifiert, és a Literalt.
Kicsit összetett, hogy mi minden épül egymásra. Ebből adódik a következő alfejezet mondandója is.

\section{Nehézségek, problémák}

\noindent

A schéma átírása során több nehézségbe is ütköztünk.
A legelső nehézség az a Visual Paradigm program korrekt használata volt.
Egy kisebb időbe tellett rájönni arra, hogy egy adott classnak hogyan kell megváltoztatni a háttérszínét, packagek közötti mozgást, hogyan kell egy adott classra hivatkozni ami másik packageben volt.
Ezután ami szerintem a legnagyobb nehézség volt, az az, hogy nem volt dokumentáció az előző schémához.
Mindent nekünk kellett kitalálni, hogy mit miért csináltak az előző fejlesztők. Akik ezt a schémát fejlesztették le, ők már nem foglalkoztak ezzel, és más nem nagyon mélyedt bele ebbe az egészbe azóta.
A következő nagyobb nehézség, az az volt, hogy összehozzunk egy olyan schémát ami működőképes, és az analyzer ezt tudja is használni.
Előző alfejezetben említettem, hogy egy adott classt (mint például ExportAllDeclaration) lefejlesszünk, ahhoz sok minden mást is le kell fejleszteni, amihez meg még több minden kellett.
Ebből adódóan tesztelni nagyon nem tudtunk, mivel az egésznek működnie kellett ahhoz, hogy az analyzer egyáltalán lefusson.
Mivel mi majdnem nulláról írtuk újra, ez volt a hátrány.
Amikor elérkeztünk ahhoz az állapothoz, hogy mindenre rámondjuk azt, hogy működik, akkor jött egy nagyobb probléma.
Valahol Segmentation Fault-ot kapott a program, és nem nagyon tudtuk ezt debugolni.
Több sejtésünk is volt, hogy mi lehet a baj. Emiatt az egész schémát át kellett nézni alaposan, hogy hol vétettünk hibákat.
Sok helyen voltak pontatlanságok, rossz osztályból származtattunk le, rossz típus volt megadva attribútumnak. Ezeket mind kijavítottuk, de most már más hibát kaptunk.
A JavaScript Analyzerben kiderítettük, hogy pontosan hol szállt el a program.
A Literal volt a hiba, mivel ezt nem a github alapján írtuk meg, hanem egyedi ötlettel. Előző schémában is máshogy volt megoldva.
\begin{figure}[!htbp]
      \caption{A DataStructures packageben a Kind package felépítése}\label{fig:literal}
      \centering
      \includegraphics[width=0.8\textwidth]{literal.png}
\end{figure}

\Aref{fig:literal} ábrán látható, hogy a LiteralExpressionből (Ami a Literal) több literal is öröklődik.
\begin{lstlisting}[caption={Literal},label={lst:asg_file_literal}, language={JavaScript}]
export interface LiteralBase extends BaseNode {
  type: AST_NODE_TYPES.Literal;
  raw: string;
  value: RegExp | bigint | boolean | number | string | null;
}
\end{lstlisting}
Próbáltuk úgy megoldani, hogy 4 vagyolással 4 különböző literál tartalmazza, de az analyzerben máshogy kezeltük le a literalt, mint nodeot.
Ezért a tartalmazás helyett inkább örököltettünk a literalból, így megoldva ezt a problémát.
Még az is probléma volt, hogy volt egy LiteralBaseünk, amiből származott ez az 5 literal. A LiteralBase származott a LiteralExpressionből, de valamiért Segmentation fault lett a vége ha így próbáltuk megoldani.
Ezért a LiteralBase-t kivettük, és LiteralExpressionből származik minden literal.
Végül még az volt a nehézség, hogy kódrészletet keressünk az analyzernek, amin le tudjuk tesztelni, hogy az analyzer jó outputot ad-e.

\noindent

Mivel nem nagyon volt typescriptes háttértudásunk, először a typescriptet kellett átnézni, hogy mit hogyan tudunk megvalósítani.
Emellett még a javascriptes outputokat is át kellett nézni, mivel a mi célunk a fejlesztés volt. A jelenlegi javascriptes projektekre ugyanolyan eredményt adott, sőt néhány helyen jobbat is, mert az előző schémában is voltak pontatlanságok.
A typescriptes projektek nagy részét tudja elemezni az analyzer, de közel sem tökéletes, mivel egyfolytában kellene fejleszteni a schémát, mivel a typescript nem annyira régi.

\chapter{JavaScriptAddon}\label{chap:JavaScriptAddon}

\section{JavaScriptAddonról}

\noindent

A JavaScriptAddon az egy node kiterjesztésű file, amit a SchemaGenerator generál. Minden egyes nodehoz külön generál header és src filet.
Ezeket a fileokat aztán mergeli, és ebből lesz a javascriptAddon.
A SchemaGenerator a JavaScriptSchemából és az ebből generált asgből generálja le a javascriptAddon.node filet.
A SchemaGeneratornak több generálása is definiálva van, hiszen nem csak erre használják. Nekünk a NodeAddonGenerator.c fileban és a hozzátartozó header fileban van minden.
A main.c-be ezt beimportálja, és ennek hívja meg egyes metódusait, ami által elkészül a javascriptAddon.
A következő pár oldalon bemutatnám, hogy a generálás hogyan zajlik.
A SchemaGeneratornak rengeteg kapcsolojó van, köztük a generateNodeAddon is. Ezekre a kapcsolókra vizsgál egyesével, egy nagy if-elseben.
\begin{lstlisting}[caption={SchemaGenerator kapcsoló vizsgálás},label={lst:schemagenerator_argv_genNodeAddon}, language={CStyle}]
else if(!strcmp(argv[i], "-genNodeAddon")){
      options.generateNodeAddon = true;
}
\end{lstlisting}
A kód elején történik meg ez, ha meg van adva paraméternek a genNodeAddon string, akkor az options.generateNodeAddon-t igazra állítja. A default értéke false.
Ezután jóval lentebb miután rengeteg mindent legenerált ami kell alapból is a normális működéshez, vizsgál egyet az options.generateNodeAddon-ra.
Ez látható \aref{lst:schemagenerator_genNodeAddon_check} kódrészleten is. Else ága nincs, szóval semmi nem történik ha nincs megadva a genNodeAddon string a paramétereknél.
\begin{lstlisting}[caption={SchemaGenerator javascriptAddon generálás},label={lst:schemagenerator_genNodeAddon_check}, language={CStyle}]
//generate Node.JS Addon if specified
if(options.generateNodeAddon ){
      debugMessage(0, "Generating Node.JS Addon sources\n");
      if (createAndEnterDirectory(SOURCE_NODE_ADDON_DIR_NAME)) {
      generatePackageJson();
      generateBindingGyp();
      generateAddonCC();
      generateFactoryWrapper();

      if (createAndEnterDirectory("inc")) {
      generateWrapperHeaders();
      leaveDirectory();}
      if(createAndEnterDirectory("src")){
      generateWrapperSources();
      leaveDirectory();}

      leaveDirectory();}
}
\end{lstlisting}

A createAndEnterDirectory metódus annyit csinál, hogy létrehoz egy mappát és chdirrel belelép. Ha az adott mappa már létezik, akkor csak szimplán belelép.
A SOURCE NODE ADDON DIR NAME változó jelen esetben addon értéket fog kapni, hiszen a javascriptAddon dolgai ebbe fognak generálódni.
Miután létrehozta és/vagy belelépett az addon mappába először legenerálja a package.json filet a projekthez.
A package.json fileban beállítja a projektnevét, verziószámát, dependencyket, scripteket és a végén a gypfile kapcsolónak egy true-t beállít.
\begin{lstlisting}[caption={NodeAddonGenerator package.json scripts}, label={lst:nodeAddonGenerator_package_json}, language={CStyle}]
fprintf(f, "    \"rebuild\": \"node-gyp configure && node-gyp rebuild -j 8\",\n");
fprintf(f, "    \"install\": \"node-gyp configure && node-gyp build -j 8\"\n");
\end{lstlisting}
\Aref{lst:nodeAddonGenerator_package_json} kódrészleten látható, hogy majd gyp segítségével fog történni a generálás.

\noindent

Ezután legenerálja a binding.gyp filet. Ez a file fog felelni azért, hogy a javascriptAddon legenerálódjon sikeresen. Ha ez megtörtént, utána fogja legenerálni az addonCC-t.
Az addon.cc fileban beimportálja a Factory.h headert, amiben több metódus is megtalálható.
\begin{lstlisting}[caption={Addon.cc Wrapperek includolása}, label={lst:addoncc_wrapper_include}, language{CStyle}]
if (!traversalDescendantBFT(rootNode, generateWrapIncludes, false)) {
      debugMessage(0, " failed\n");
      fclose(f);
      return false;
}
\end{lstlisting}

\Aref{lst:addoncc_wrapper_include} kódrészletben a traversalDescendantBFT metódus a Factory.h-ban lett létrehozva.
Ez egy bejárás, ami az összes nodera lefut, és az összes nodera meghívja a generateWrapIncludes metódust.
Ez a generateWrapIncludes a nodeAddonGenerator.c fileban van leimplementálva a következőképp:
\begin{lstlisting}[caption={generateWrapIncludes leimplementálása}, label={lst:addoncc_wrapper_includes_implementation}, language{CStyle}]
if(node->type.abstract){
      return true;
}
fprintf(f, "#include \"");
fprintf(f, "inc/%sWrapper.h\"\n", node->name );
return true;
\end{lstlisting}
Először megvizsgálja, hogy az adott node abstract-e, ha nem akkor tovább megy, és az addon.cc-be includolja az adott wrappert.
A node->name lehet például FunctionDeclaration, ClassDeclaration, amik megtalálhatóak az asg fileban.

Ezután ezt a bejárást mégegyszer végrehajtja, de most a wrapperIniteket generálja az addon.cc fileba.
Annyi változtatással, hogy picit mást ír az addon.cc fileba.
\begin{lstlisting}[caption={generateWrapInit leimplementálása}, label={addoncc_wrapper_inits_implementation}, language={CStyle}]
fprintf(f, "  columbus::%s::asg::addon::%sWrapper::Init(env, exports);\n", schemaName, node->name);
\end{lstlisting}

Ha ezek megtörténtek, akkor az addon.cc filet legeneráltuk sikeresen, és így benne találhatóak a wrapperek includolása és a wrapperek Initjei.

\noindent

Ezután következett a generateFactoryWrapper. A generateFactoryWrapperben 2 metódus hivás található, a generateFactoryWrapperHeader és generateFactoryWrapperSource.
A generateFactoryWrapperHeader a Factory.h filet generálta le, a generateFactoryWrapperSource pedig a Factory.cc filet.
Ebben a Factory.cc fileban található az összes metódus, amit fog használni a javascriptAddon.
Emellett az összes nodeWrapper includeolva van, amit ezután fog létrehozni.

\noindent

Utolsó lépésként, először az inc mappát majd az src mappát generálja le a SchemaGenerator.
Az inc mappában találhatóak a header fileok egyes wrappereknek, az src mappában maguk a wrapperek vannak megvalósítva.
\todoi{Megnézni egy wrapper megvalósítást bent és leírni.}
%generateFactoryWrapper
%mi ez a külön file
% mi ez a factory, hogyan használja a JSAN
%-mit ad factory, mi ez a külön file
%-Miben változott
% Megemlíteni, hogy a main.cppben hogy generálja le az addont
\section{Miben változott}

\chapter{JSAN2Lim átírása}\label{chap:JSAN2Lim átírása}

\section{JSAN2Limről}
%Leírni, hogy a JSAN2Lim hogy alakít át limre, miért fontos nekünk a lim, JSAN outputról pár szót
%Esetleg a halstead-ot is megemlíteni
\section{Bővítések}
%Leírni, hogy typescriptet nem detektált, ezel kellett kibővíteni, talán ide néhány kódrészletet

%-halstead, stb
%-miben kellett bőviteni, mi hiányzott
%-mit nem detektált
\chapter*{AST Binder Optimalizálás}\stepcounter{chapter}
\addcontentsline{toc}{chapter}{AST Binder Optimalizálás}
-node addon mi
-miért lassú
-javítás, ha lehet, vagy ha nem, miért
- nehézségek, eredmények 
% \chapter{AST Binder Optimalizálás}\label{chap:AST Binder Optimalizálás}

\section{Binderről}

\section{Lassúság okai}

\section{Optimalizálás}

\section{Eredmény}


\chapter{Regteszt frissítés}\label{chap:Regteszt_frissítés}

\section{A regressziós tesztelésről}

\noindent

A Sourcemeter Javascript projektben regressziós tesztelés folyik, mint tesztelési folyamat.
A regressziós teszt segítségével hamar tudunk hibákat kiszűrni fejlesztés során.
Ez a következőképpen zajlik a projekten:

\begin{itemize}
      \item Cmake segítségével először legeneráltatjuk a megfelelő fileokat. Vcxproj illetve make fileokat, használt operációs rendszertől függően.
      \item Visual Studio 2017 vagy make segítségével lebuildeljük a Regtest\_javascript targetet.
      \item A Regtest\_javascript buildelése során ellenőrizzük, hogy az adott projekt le van-e már buildelve. Ha nem, akkor lebuildeljük tesztelés előtt.
      \item Tesztelés közben, programonként írja ki a konzolra, hogy sikeres vagy sem az adott teszt.
      \item Tesztelés végén egy regtest.xml fileba írja az összesített eredményeket fileokra lebontva.
      \item Ha valami differencia van a referenciához képest, akkor azt egy külön diff kiterjesztésű fileban jelzi a rendszer nekünk, az elvárt és a kapott eredményt beleírva.
\end{itemize}

\section{Tesztekről}

\noindent

A Regtest\_javascript target több programot foglal magában, nem csak a JSAN-t. Egészen pontosan a következőket:
JSAN, JSAN2Lim, LIM2Metrics, LIM2Patterns, ESLintrunner, ESLint2Graph, ChangeTracker, DuplicatedCodefinder és SourceMeter projekteknek tartalmazza a tesztjeit javascriptes (és mostmár typescriptes) fileokra.
Én a JSAN, JSAN2Lim, ESLintrunner és a SourceMeter projekteknek változtattam a tesztelésén.
Eddig a tesztelés úgy zajlott, hogy beadtunk inputnak pár nagyobb filet, és lefuttattuk rá a programot amit tesztelni szerettünk volna, és megnéztük az outputot.
Azt kaptam feladatnak, hogy a tesztelési logikát írjam át arra, hogy kapcsolókra is teszteljünk.
Ez azt jelentette, hogy minden kisebb projektnek vannak külön kapcsolói. Eddig mindig csak egy adott sorral futtattuk le a programot, és nem volt az tesztelve, hogy pl useRelativePath működik-e az elvártak szerint vagy sem.

\section{Tesztek kibővítése}

\noindent

Először kezdtem a JSAN tesztek átírásával. A JSAN-nak a következő kapcsolói voltak:
\begin{itemize}
      \item -i:Jelentése input, lehet relatív vagy abszolút útvonal a filehoz vagy projekthez.
      \item -o:Jelentése output neve, lehet relatív vagy abszolút útvonal.
      \item -d:Jelentése dumpjsml, a JSAN outputját átgenerálja XML stílusú fileba és ezt egy jsml fileba kiírja.
      \item -e:Jelentése ExternalHardFilter, relatív vagy abszolút útvonal egy olyan filehoz, ami szövegalapú és olyan syntax található benne, ami kell az externalHardFilternek
      \item -help:Jelentése help, kiírja minden kapcsolóhoz tartozó descriptiont.
      \item -r:Jelentése useRelativePath, outputban az útvonalakat átírja relatív útvonalra.
      \item -h:Jelentése html, a JSAN html fileokra is lefut, bennük keresve javascriptes scripteket és azokat tesztelni.
      \item -stat:Jelentése statistics, Kiírja a memóriahasználatot és a futásidőt amit a JSAN vett igénybe.
\end{itemize}

\noindent

Az input, output, dumpjsml, ExternalHardFilter, és html kapcsolókra tudtam tesztelést írni. A logika az volt, hogy mappanév alapján tesztelek egyes kapcsolókra.
ProgramozásiNyelv-kapcsolónév logikát követtem, mivel javascriptes tesztek mellé kell majd typescriptes teszteket is keresnem.
A projekteket python scriptek segítségével futtattam le.

\begin{lstlisting}[caption={JSAN kapcsoló vizsgálat pythonban}, label={lst:python_kapcsolo}, language={Python}]
if "externalHardFilter" in input_path:
      ret_val = self._execute_one_test(input_path, external_hard_filter=True) and ret_val
      return ret_val
\end{lstlisting}

\Aref{lst:python_kapcsolo} kódrészleten látható, hogy hogyan keresek egy adott kapcsolóra. Az input\_pathban van a mappa is, és ugye a js-externalHardFilter-ben megtalálható az externalHardFilter szó.
Az execute\_one\_test függvényemben beállítom a teszteléshez az adott dolgokat.

\begin{lstlisting}[caption={JSAN kapcsoló beállítása pythonban}, label={lst:python_kapcsolo_beallitasa}, language={Python}]
if external_hard_filter:
      input_dir = os.path.dirname(input_path)
      external_hard_filter_path = os.path.join(input_dir, "externalHardFilter.txt")
      external_hard_filter_switch = "-e"
else:
      external_hard_filter_path = ""
      external_hard_filter_switch = ""
\end{lstlisting}

\Aref{lst:python_kapcsolo_beallitasa} kódrészletben az execute\_one\_test függvénynek egy részét láthatjuk, ahol beállítjuk az external\_hard\_filter\_path-t és a switchet annak függvényében, hogy igazat kaptunk e vagy sem.
Utána kellett néznem, hogy egy ExternalHardFilter file hogy is néz ki, hogyan kell használni.
A használata a következő: létrehozzuk az externalHardFilter filet a tesztelendő file mellé vagy a tesztelendő projekt gyökerébe,
és attól függően, hogy ki akarjuk hagyni az adott filet vagy hozzáadni, írunk egy $+$ vagy egy $-$ jelet a sor elejére és utána relatív útvonal és a file neve.
Alapértelmezetten minden filet leelemez az adott program, JSAN esetében ezek azok a fileok amiknek a kiterjesztése js,jsx. (külön kapcsolóval megadhatjuk neki, hogy a html kiterjesztésű fileokat elemezze-e vagy se.)
Typescriptes bővítés után már a ts és a tsx kiterjesztésű fileokat is elemzi. Ezért írtam több tesztesetet is. Egy példa, hogy hogyan néz ki ez a file:

\begin{lstlisting}[caption={ExternalHardFilter file}, label={lst:external_hard_filter}]
-filtered01
-filtered02
-filtered03
+filtered01
\end{lstlisting}

\Aref{lst:external_hard_filter} kódrészletben látható, hogy filtered01 02 és 03 at kivettük, hogy azokat ne elemezze a jsan.
Ezután visszavettük a filtered01et. A filtered fileok azok javascriptes fileok, javascriptes kóddal.
Ezután referenciába csak a filtered01 file outputját raktam be, hiszen a 02 és 03at nem elemzi, ha elemezné, akkor szólna a program, hogy missing reference file.
Természetesen lehet regexpet is használni filterezésnél.

\noindent

Ezután teszteltem a $-i$ kapcsolót, itt ugye a programnak vissza kellene adnia, hogy üres az input ha nincs megadva.
Ezután a $-o$ kapcsolóra teszteltem, itt ebben az esetben default értékben $out.jssi$ filet kellene visszaadnia.
Végül a $-h$ kapcsolót néztem meg, itt ha megvan adva ez a kapcsoló, akkor az adott projektben a html fileokban a javascriptet kellene tesztelni.
A $-d$, $-help$, $-stat$ kapcsolókra nem tudtam tesztelni, még a $-useRelativePath$ kapcsolóra sem, hiszen ha abszolút utat kérek, akkor a referenciákban másnál rossz lesz az elvárt eredmény.

\noindent

Ezután a JSAN2Lim és az ESlintrunner programoknak írtam át a tesztelési menetetét, mivel nekik volt még olyan kapcsoló, amit lehetett tesztelni. A többi projektnek 1 vagy 2 volt, amik kellettek a működésükhöz, nem voltak opcionális kapcsolók.

\noindent

Miután a JSAN ki lett egészítve typescriptes supportal és a JSAN2Limet átírtam, hogy a JSAN általi typescript elemzéseket jól olvassa be, ideje volt teszteket keresni, mind JSAN-nak és mind JSAN2Limnek.
Kerestem egyszerűbb typescript és picit összetettebb typescriptes projekteket is, hogy lássam a hiányosságokat.
Természetesen ami outputokat adtak a programok, azokat át kellett néznem egyesével, hiszen csak így tudom meg, hogy jól tesztelte-e a megadott filet vagy sem.
Több hiányosságot is észrevettem, mind JSAN oldalról, mind JSAN2Lim oldalról, ezek kisebb hiányosságok voltak.
Például, hogy egy nodenak nem volt beállítva pozíció, vagy nem volt jól beállítva a paramétere.

\chapter{Összefoglaló}\label{chap:Összefoglaló}

\section{Program javulása}

\noindent

A szakdolgozatom során sikerült elérnem azt, hogy a JavaScript-es nyelvi séma tartalmazza a TypeScript-es nyelvi elemeket is.
Továbbá a JSAN2Lim-et sikerült kibővíteni TypeScript támogatással.
Mindeközben a regressziós teszteléseket sikeresen frissítettem és validáltam, amit később mások is jóvá hagytak.
Legvégül sikerült az AST binder metódusát optimalizálnom, sokkal gyorsabban fut le, mint ezelőtt.

\section{Jövőbeli tervek}

\noindent

Rengeteg változtatás volt a TypeScript résznél, de a sémát nem tudtuk még naprakészre hozni, mivel voltak ennél fontosabb feladatok.
Mint például az AST binder optimalizálása, vagy a JSAN2Lim továbbfejlesztése.
Ebből kifolyólag jelenleg a séma nem naprakész.
Mivel a szakdolgozatom elkészítése során megfelelő jártasságra tettem szert, így a későbbi bővítést az eddigiektől gyorsabban el tudnám végezni.

\include{tex/Nyilatkozat}
\addcontentsline{toc}{chapter}{Irodalomjegyzék}
\bibliographystyle{plain}
\bibliography{references}

\end{document}
