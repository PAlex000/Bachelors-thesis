\documentclass[12pt,a4paper]{report}
\usepackage[margin=2.5cm]{geometry}
\usepackage[magyar]{babel}

% magyar nyelv tamogatas
\usepackage{t1enc}
\usepackage[T1]{fontenc}
\usepackage[utf8]{inputenc}

% A formai kovetelmenyekben megkövetelt Times betűtípus hasznalata:
\usepackage{times}

\usepackage{setspace}
\usepackage{listings,multicol}
\usepackage{xcolor}
\usepackage{textcomp}
\usepackage{enumitem}
\usepackage{booktabs}
\usepackage[unicode,hidelinks]{hyperref}
\usepackage{footnote}
\usepackage{ifthen}

% Törölhető package
\usepackage{lipsum}

% TODO csomag, amivel jól észrevehető todokat hagyhatunk a dolgozatban
\usepackage{todonotes}
% Inline TODO
\newcommand{\todoi}[1]{\todo[inline]{\textbf{TODO:} #1}}

% egyedi lablec
\usepackage{fancyhdr}
\usepackage{graphicx}
\graphicspath{{fig/}}

% Kódrészletes színezése
\definecolor{lightgray}{rgb}{.9,.9,.9}
\definecolor{darkgray}{rgb}{.4,.4,.4}
\definecolor{purple}{rgb}{0.65, 0.12, 0.82}

\lstdefinelanguage{JavaScript}{
  keywords={typeof, new, true, false, catch, function, return, null, catch, switch, var, if, in, while, do, else, case, break},
  keywordstyle=\color{blue}\bfseries,
  ndkeywords={class, export, boolean, throw, implements, import, this},
  ndkeywordstyle=\color{darkgray}\bfseries,
  identifierstyle=\color{black},
  sensitive=false,
  comment=[l]{//},
  morecomment=[s]{/*}{*/},
  commentstyle=\color{purple}\ttfamily,
  stringstyle=\color{red}\ttfamily,
  morestring=[b]',
  morestring=[b]"
}

\lstset{
   language=JavaScript,
   backgroundcolor=\color{lightgray},
   extendedchars=true,
   basicstyle=\footnotesize\ttfamily,
   showstringspaces=false,
   showspaces=false,
   numbers=left,
   numberstyle=\footnotesize,
   numbersep=9pt,
   tabsize=2,
   breaklines=true,
   showtabs=false,
   captionpos=b
}

\lstdefinelanguage{CStyle}{
    backgroundcolor=\color{lightgray},   
    commentstyle=\color{purple}\ttfamily,
    keywordstyle=\color{blue}\bfseries,
    numberstyle=\footnotesize,
    stringstyle=\color{red}\ttfamily,
    basicstyle=\footnotesize\ttfamily,
    breakatwhitespace=false,         
    breaklines=true,                 
    captionpos=b,                    
    keepspaces=true,                 
    numbers=left,                    
    numbersep=9pt,                  
    showspaces=false,                
    showstringspaces=false,
    showtabs=false,                  
    tabsize=2,
    language=C
}

\renewcommand{\lstlistingname}{Kódrészlet}

% Margók beállítása
\hoffset -1in
\voffset -1in
\oddsidemargin 35mm
\textwidth 150mm
\topmargin 15mm
\headheight 10mm
\headsep 5mm
\textheight 237mm

% Szerző és dolgozat adatai
%Szerző adatai
\newcommand{\nev}{Pozsgai Alex}
\newcommand{\szak}{programtervező informatikus BSc}
\newcommand{\tanszek}{Szoftverfejlesztés}
\newcommand{\ev}{2023}
\newcommand{\dolgozatTipusa}{Szakdolgozat}
\newcommand{\vegsoDatum}{\today}

\newcommand{\cim}{Ipari JavaScript elemző kiegészítése TypeScript támogatással}
\newcommand{\angolcim}{Industrial JavaScript analyzer enhancement with TypeScript support}

%Témavezető adatai
\newcommand{\temavezetoNev}{Dr.Antal Gábor}
\newcommand{\temavezetoBeosztas}{Tudományos munkatárs}


\begin{document}

% Másfeles sorköz
\setstretch{1.5}
\sloppy

\thispagestyle{empty}
\pagenumbering{gobble}
\begin{center}
  \vspace*{0.5cm}
  {
    \Large\bf Szegedi Tudományegyetem}

    \vspace{0.1cm}

    {\Large\bf Informatikai Intézet}

    \vspace*{4.2cm}

    {\LARGE\bf \cim}

    \vspace*{1.4cm}

    {\Large \dolgozatTipusa}

    \vspace*{3.5cm}

    %Értelemszerűen megváltoztatandó:
    {\large
    \noindent
    \begin{tabular}{@{}c@{\hspace{1cm}}c}
    \emph{Készítette:}     & \emph{Témavezető:}\\
    \bf{\nev}              & \bf{\temavezetoNev}\\
    \szak                  & \temavezetoBeosztas\\
    szakos hallgató        &
    \end{tabular}
    }

    \vspace*{2.3cm}

    {\Large
    Szeged
    \\
    \vspace{2mm}
    \ev
  }
\end{center}

\pagenumbering{arabic}
\chapter*{Feladatkiírás}
\addcontentsline{toc}{section}{Feladatkiírás}
\todoi{A feladatkiírás befejezése}
\include{tex/0_tartalmi}
\addcontentsline{toc}{section}{Tartalomjegyzék}

\thispagestyle{plain}
\tableofcontents

\pagestyle{fancy}
\fancyhf{}
\fancyhead[L]{\textit{\cim}}
\fancyfoot[R]{\thepage}
\fancypagestyle{plain}{%
    \renewcommand{\headrulewidth}{0pt}%
    \fancyhf{}%
    \fancyfoot[R]{\thepage}%
}
\chapter{Bevezetés}
\label{chap:intro}

\todoi{Bevezetés befejezése}

\section{Egy alfejezet}

\section{Még egy alfejezet}

Egy hivatkozás~\cite{Martin:2008:CCH:1388398, gerrit, pep8}.

\section{Harmadik alfejezet}

% \begin{figure}[!htbp]
% \caption{Egy nagyon szép kép}
% \label{fig:szepkep}
% \centering
% \includegraphics[width=1.0\textwidth]{szep_kep.png}
% \end{figure}

% \Aref{fig:szepkep}. ábra, és \aref{fig:szepkep}. ábra.

\chapter{Még egy fejezet}
\label{chap:fejezet2}

\begin{lstlisting}[caption={Hogyan ne írjunk kódot},label={lst:stringstartswith}, language={Python}]
szoveg = "valamiszoveg"
if szoveg[:3] == "val":
    print("igaz")
else:
    print("hamis")
\end{lstlisting}

\begin{lstlisting}[caption={Miből lehet probléma},label={lst:stringstartswith2}, language={Python}]
szoveg = "valamiszoveg"
if szoveg[:3] == "valami":
    print("igaz")
else:
    print("hamis")
\end{lstlisting}

\Aref{lst:stringstartswith}. kódrészlet és \aref{lst:stringstartswith}. kódrészlet.
\chapter{Összefoglaló}
\label{chap:conclusion}

\chapter*{Nyilatkozat}
\addcontentsline{toc}{chapter}{Nyilatkozat}

% A szövegben a dolgozat típusa alapján "diplomamunkámat" vagy "szakdolgozatomat"
% szövegrészt készíti el. 
\ifthenelse{\equal{\dolgozatTipusa}{Diplomamunka}}
  {% True case
   \newcommand{\dolgozatomat}{diplomamunkámat}%
  }
  {% false case
   \newcommand{\dolgozatomat}{szakdolgozatomat}%
  }
%

\noindent Alulírott \nev{} \szak{} szakos hallgató, kijelentem, hogy a dolgozatomat a Szegedi Tudományegyetem, Informatikai Intézet \tanszek{} Tanszékén készítettem, \szak{} diploma megszerzése érdekében. 

Kijelentem, hogy a dolgozatot más szakon korábban nem védtem meg, saját munkám eredménye, és csak a hivatkozott forrásokat (szakirodalom, eszközök, stb.) használtam fel. 

Tudomásul veszem, hogy \dolgozatomat{} a Szegedi Tudományegyetem Informatikai Intézet könyvtárában, a helyben olvasható könyvek között helyezik el.

\vspace*{2cm}

\begin{table}[!h]
  \begin{tabular}{lc}
    Szeged, \vegsoDatum{} \hspace{2cm}  & \makebox[7cm]{\dotfill} \\
                                        & aláírás \\
  \end{tabular}
\end{table}

\thispagestyle{plain}
\include{tex/acknowledgement}
\addcontentsline{toc}{chapter}{Irodalomjegyzék}
\bibliographystyle{plain}
\bibliography{references}

\end{document}
