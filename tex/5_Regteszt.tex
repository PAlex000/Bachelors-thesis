\chapter{Regteszt frissítés}\label{chap:Regteszt frissítés}

\section{A regressziós tesztelésről}

\noindent

A Sourcemeter Javascript projektben regressziós tesztelés folyik, mint tesztelési folyamat.
A regressziós teszt segítségével hamar tudunk hibákat kiszűrni fejlesztés során.
Ez a következőképpen zajlik a projekten:

\begin{itemize}
      \item Cmake segítségével először legeneráltatjuk a megfelelő fileokat. Vcxproj illetve make fileokat, használt operációs rendszertől függően.
      \item Visual Studio 2017 vagy make segítségével lebuildeljük a Regtest\_javascript targetet.
      \item A Regtest\_javascript buildelése során ellenőrizzük, hogy az adott projekt le van-e már buildelve. Ha nem, akkor lebuildeljük tesztelés előtt.
      \item Tesztelés közben, programonként írja ki a konzolra, hogy sikeres vagy sem az adott teszt.
      \item Tesztelés végén egy regtest.xml fileba írja az összesített eredményeket fileokra lebontva.
      \item Ha valami differencia van a referenciához képest, akkor azt egy külön diff kiterjesztésű fileban jelzi a rendszer nekünk, az elvárt és a kapott eredményt beleírva.
\end{itemize}

\section{Tesztekről}

\noindent

A Regtest\_javascript target több programot foglal magában, nem csak a JSAN-t. Egészen pontosan a következőket:
JSAN, JSAN2Lim, LIM2Metrics, LIM2Patterns, ESLintrunner, ESLint2Graph, ChangeTracker, DuplicatedCodefinder és SourceMeter projekteknek tartalmazza a tesztjeit javascriptes (és mostmár typescriptes) fileokra.
Én a JSAN, JSAN2Lim, ESLintrunner és a SourceMeter projekteknek változtattam a tesztelésén.
Eddig a tesztelés úgy zajlott, hogy beadtunk inputnak pár nagyobb filet, és lefuttattuk rá a programot amit tesztelni szerettünk volna, és megnéztük az outputot.
Azt kaptam feladatnak, hogy a tesztelési logikát írjam át arra, hogy kapcsolókra is teszteljünk.
Ez azt jelentette, hogy minden kisebb projektnek vannak külön kapcsolói. Eddig mindig csak egy adott sorral futtattuk le a programot, és nem volt az tesztelve, hogy pl useRelativePath működik-e az elvártak szerint vagy sem.

\section{Tesztek kibővítése}

\noindent

Először kezdtem a JSAN tesztek átírásával. A JSAN-nak a következő kapcsolói voltak:
\begin{itemize}
      \item -i:Jelentése input, lehet relatív vagy abszolút útvonal a filehoz vagy projekthez.
      \item -o:Jelentése output neve, lehet relatív vagy abszolút útvonal.
      \item -d:Jelentése dumpjsml, a JSAN outputját átgenerálja XML stílusú fileba és ezt egy jsml fileba kiírja.
      \item -e:Jelentése ExternalHardFilter, relatív vagy abszolút útvonal egy olyan filehoz, ami szövegalapú és olyan syntax található benne, ami kell az externalHardFilternek
      \item -help:Jelentése help, kiírja minden kapcsolóhoz tartozó descriptiont.
      \item -r:Jelentése useRelativePath, outputban az útvonalakat átírja relatív útvonalra.
      \item -h:Jelentése html, a JSAN html fileokra is lefut, bennük keresve javascriptes scripteket és azokat tesztelni.
      \item -stat:Jelentése statistics, Kiírja a memóriahasználatot és a futásidőt amit a JSAN vett igénybe.
\end{itemize}

\noindent

Az input, output, dumpjsml, ExternalHardFilter, és html kapcsolókra tudtam tesztelést írni. A logika az volt, hogy mappanév alapján tesztelek egyes kapcsolókra.
ProgramozásiNyelv-kapcsolónév logikát követtem, mivel javascriptes tesztek mellé kell majd typescriptes teszteket is keresnem.
A projekteket python scriptek segítségével futtattam le.

\begin{lstlisting}[caption={JSAN kapcsoló vizsgálat pythonban}, label={lst:python_kapcsolo}, language={Python}]
if "externalHardFilter" in input_path:
      ret_val = self._execute_one_test(input_path, external_hard_filter=True) and ret_val
      return ret_val
\end{lstlisting}

\Aref{lst:python_kapcsolo} kódrészleten látható, hogy hogyan keresek egy adott kapcsolóra. Az input\_pathban van a mappa is, és ugye a js-externalHardFilter-ben megtalálható az externalHardFilter szó.
Az execute\_one\_test függvényemben beállítom a teszteléshez az adott dolgokat.

\begin{lstlisting}[caption={JSAN kapcsoló beállítása pythonban}, label={lst:python_kapcsolo_beallitasa}, language={Python}]
if external_hard_filter:
      input_dir = os.path.dirname(input_path)
      external_hard_filter_path = os.path.join(input_dir, "externalHardFilter.txt")
      external_hard_filter_switch = "-e"
else:
      external_hard_filter_path = ""
      external_hard_filter_switch = ""
\end{lstlisting}

\Aref{lst:python_kapcsolo_beallitasa} kódrészletben az execute\_one\_test függvénynek egy részét láthatjuk, ahol beállítjuk az external\_hard\_filter\_path-t és a switchet annak függvényében, hogy igazat kaptunk e vagy sem.
Utána kellett néznem, hogy egy ExternalHardFilter file hogy is néz ki, hogyan kell használni.
A használata a következő: létrehozzuk az externalHardFilter filet a tesztelendő file mellé vagy a tesztelendő projekt gyökerébe,
és attól függően, hogy ki akarjuk hagyni az adott filet vagy hozzáadni, írunk egy $+$ vagy egy $-$ jelet a sor elejére és utána relatív útvonal és a file neve.
Alapértelmezetten minden filet leelemez az adott program, JSAN esetében ezek azok a fileok amiknek a kiterjesztése js,jsx. (külön kapcsolóval megadhatjuk neki, hogy a html kiterjesztésű fileokat elemezze-e vagy se.)
Typescriptes bővítés után már a ts és a tsx kiterjesztésű fileokat is elemzi. Ezért írtam több tesztesetet is. Egy példa, hogy hogyan néz ki ez a file:

\begin{lstlisting}[caption={ExternalHardFilter file}, label={lst:external_hard_filter}]
-filtered01
-filtered02
-filtered03
+filtered01
\end{lstlisting}

\Aref{lst:external_hard_filter} kódrészletben látható, hogy filtered01 02 és 03 at kivettük, hogy azokat ne elemezze a jsan.
Ezután visszavettük a filtered01et. A filtered fileok azok javascriptes fileok, javascriptes kóddal.
Ezután referenciába csak a filtered01 file outputját raktam be, hiszen a 02 és 03at nem elemzi, ha elemezné, akkor szólna a program, hogy missing reference file.
Természetesen lehet regexpet is használni filterezésnél.

\noindent

Ezután teszteltem a $-i$ kapcsolót, itt ugye a programnak vissza kellene adnia, hogy üres az input ha nincs megadva.
Ezután a $-o$ kapcsolóra teszteltem, itt ebben az esetben default értékben $out.jssi$ filet kellene visszaadnia.
Végül a $-h$ kapcsolót néztem meg, itt ha megvan adva ez a kapcsoló, akkor az adott projektben a html fileokban a javascriptet kellene tesztelni.
A $-d$, $-help$, $-stat$ kapcsolókra nem tudtam tesztelni, még a $-useRelativePath$ kapcsolóra sem, hiszen ha abszolút utat kérek, akkor a referenciákban másnál rossz lesz az elvárt eredmény.

\noindent

Ezután a JSAN2Lim és az ESlintrunner programoknak írtam át a tesztelési menetetét, mivel nekik volt még olyan kapcsoló, amit lehetett tesztelni. A többi projektnek 1 vagy 2 volt, amik kellettek a működésükhöz, nem voltak opcionális kapcsolók.

\noindent

Miután a JSAN ki lett egészítve typescriptes supportal és a JSAN2Limet átírtam, hogy a JSAN általi typescript elemzéseket jól olvassa be, ideje volt teszteket keresni, mind JSAN-nak és mind JSAN2Limnek.
Kerestem egyszerűbb typescript és picit összetettebb typescriptes projekteket is, hogy lássam a hiányosságokat.
Természetesen ami outputokat adtak a programok, azokat át kellett néznem egyesével, hiszen csak így tudom meg, hogy jól tesztelte-e a megadott filet vagy sem.
Több hiányosságot is észrevettem, mind JSAN oldalról, mind JSAN2Lim oldalról, ezek kisebb hiányosságok voltak.
Például, hogy egy nodenak nem volt beállítva pozíció, vagy nem volt jól beállítva a paramétere.
