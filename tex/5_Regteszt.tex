\chapter{Regressziós tesztek frissítése}\label{chap:Regteszt_frissítés}

\section{Tesztek kibővítése}
Először kezdtem a JSAN tesztek átírásával. A JSAN-nak a következő kapcsolói voltak:
\begin{itemize}
      \item -i: Jelentése input, lehet relatív vagy abszolút útvonal a fáljoz vagy projekthez.
      \item -o: Jelentése output neve, lehet relatív vagy abszolút útvonal.
      \item -d: Jelentése dumpjsml, a JSAN eredményét átgenerálja XML stílusú fájlba és ezt egy jsml fájlba kiírja.
      \item -e: Jelentése ExternalHardFilter, relatív vagy abszolút útvonal egy olyan fájlhoz, ami szövegalapú és olyan szintaxis található benne, ami kell az externalHardFilternek.
      \item -help: Kiírja minden kapcsolóhoz tartozó leírást.
      \item -r: Jelentése useRelativePath, eredményben az útvonalakat átírja relatív útvonalra.
      \item -h: Jelentése html, a JSAN html fájlokra is lefut, bennük keresve JavaScript-es szkripteket és azokat teszteli.
      \item -stat: Jelentése statistics, kiírja a memóriahasználatot és a futásidőt amit a JSAN vett igénybe.
\end{itemize}


Az input, output, dumpjsml, ExternalHardFilter, és html kapcsolókra tudtam tesztelést írni. A logika az volt, hogy mappanév alapján tesztelek egyes kapcsolókra.
ProgramozásiNyelv-kapcsolónév logikát követtem, mivel JavaScript-es tesztek mellé kell majd TypeScript-es teszteket is keresnem.
A projekteket python szkriptek segítségével futtattam le.

\begin{lstlisting}[caption={JSAN kapcsoló vizsgálat pythonban}, label={lst:python_kapcsolo}, language={Python}]
if "externalHardFilter" in input_path:
      ret_val = self._execute_one_test(input_path, external_hard_filter=True) and ret_val
      return ret_val
\end{lstlisting}

\Aref{lst:python_kapcsolo}-es kódrészleten látható, hogy hogyan keresek egy adott kapcsolóra. Az input\_path-ban van a mappa is, és a js-externalHardFilter-ben megtalálható az externalHardFilter szó.
Az execute\_one\_test függvényemben beállítom a teszteléshez az adott dolgokat.

\begin{lstlisting}[caption={JSAN kapcsoló beállítása pythonban}, label={lst:python_kapcsolo_beallitasa}, language={Python}]
if external_hard_filter:
      input_dir = os.path.dirname(input_path)
      external_hard_filter_path = os.path.join(input_dir, "externalHardFilter.txt")
      external_hard_filter_switch = "-e"
else:
      external_hard_filter_path = ""
      external_hard_filter_switch = ""
\end{lstlisting}

\Aref{lst:python_kapcsolo_beallitasa}-es kódrészletben az execute\_one\_test függvénynek egy részét láthatjuk, ahol beállítjuk az external\_hard\_filter\_path-t és a kapcsolót annak függvényében, hogy igazat kaptunk e vagy sem.
Létrehozzuk az externalHardFilter fájlt a tesztelendő fájl mellé vagy a tesztelendő projekt gyökerébe,
és attól függően, hogy ki akarjuk hagyni az adott fájlt vagy hozzáadni, írunk egy $+$ vagy egy $-$ jelet a sor elejére és utána relatív útvonal és a fájl neve.
Alapértelmezetten minden fájl elemez az adott program, JSAN esetében ezek azok a fájlok amiknek a kiterjesztése \texttt{js}, \texttt{jsx}. (külön kapcsolóval megadhatjuk neki, hogy a html kiterjesztésű fájlokat elemezze-e vagy se.)
TypeScript-es bővítés után már a ts és a tsx kiterjesztésű fájlokat is elemzi. Ezért írtam több tesztesetet is. Egy példa, hogy hogyan néz ki ez a fájl:

\begin{lstlisting}[caption={ExternalHardFilter fájl}, label={lst:external_hard_filter}]
-filtered01
-filtered02
-filtered03
+filtered01
\end{lstlisting}

\Aref{lst:external_hard_filter}-as kódrészletben látható, hogy filtered01, filtered02 és filtered03 at kivettük, hogy azokat ne elemezze a jsan.
Ezután visszavettük a filtered01-et. A filtered fájlok azok JavaScript-es fájlok, JavaScript-es kóddal.
Ezután referenciába csak a filtered01 fájl eredményét raktam be, hiszen a 02 és 03-at nem elemzi, ha elemezné, akkor szólna a program, hogy hiányzó referecia fájl.
Természetesen lehet regurális kifejezést is használni filterezésnél.


Ezután teszteltem a \texttt{-i} kapcsolót, itt a programnak vissza kellene adnia, hogy üres az input ha nincs megadva.
Ezután a \texttt{-o} kapcsolóra teszteltem, itt ebben az esetben alapértelmezett értékben \texttt{out.jssi} fájlt kellene visszaadnia.
Végül a \texttt{-h} kapcsolót néztem meg, itt ha megvan adva ez a kapcsoló, akkor az adott projektben a html fájlokban a JavaScript-et kellene tesztelni.
A \texttt{-d}, \texttt{-help}, \texttt{-stat} kapcsolókra nem tudtam tesztelni, még a \texttt{-useRelativePath} kapcsolóra sem, hiszen ha abszolút utat kérek, akkor a referenciákban másnál rossz lesz az elvárt eredmény.


Ezután a JSAN2Lim és az ESLintrunner programoknak írtam át a tesztelési menetetét, mivel nekik volt még olyan kapcsoló, amit lehetett tesztelni. A többi projektnek 1 vagy 2 volt, amik kellettek a működésükhöz, nem voltak opcionális kapcsolók.

Miután a JSAN ki lett egészítve TypeScript támogatással és a JSAN2Limet átírtam, elkezdtem teszteket keresni, mind a JSAN-nak és mind JSAN2Lim-nek.
Kerestem egyszerűbb TypeScript és picit összetettebb TypeScript projekteket is, hogy lássam az esetleges hiányosságokat.
Természetesen ami eredményeket adtak a programok, azokat át kellett néznem egyesével, hiszen csak így tudom meg, hogy jól tesztelte-e a megadott fájlt vagy sem.
Több kisebb hiányosságot is észrevettem, mind JSAN oldalról, mind JSAN2Lim oldalról, ezek kisebb hiányosságok voltak.
Például, hogy egy node-nak nem volt beállítva pozíció, vagy nem volt jól beállítva a paramétere.
