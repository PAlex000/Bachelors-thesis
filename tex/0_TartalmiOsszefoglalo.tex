\chapter*{Tartalmi összefoglaló}
\addcontentsline{toc}{section}{Tartalmi összefoglaló}

\noindent\textbf{A téma megnevezése:}

JavaScript Analyzer kiegészítése TypeScript supporttal, JSAN optimalizálása.

\noindent\textbf{A megadott feladat megfogalmazása:}

A feladat során el kell érni azt, hogy a JavaScript Analyzer Tool tudjon TypeScript fileokra lefutni, és azokat nagy pontossággal elemezni. 
Emellett a JSAN működését optimalizálni.

\noindent\textbf{A megoldási mód:}

A megoldás során kell változtatni a JavaScriptSchémán, JSAN programban azon belül is az astTransformer.js fileban és a JSAN2Limben.

\noindent\textbf{Alkalmazott eszközök, módszerek:}

A megoldáshoz Visual Studio Code-t, a JavaScriptSchema szerkesztéséhez Visual paradigm-t használtam.
A projekt lebuildeléséhez és ellenőrzéséhez Visual Studio 2017 programot használtam.

\noindent\textbf{Elért eredmények:}

JSAN képes TypeScript fileokat nagy pontossággal elemezni, nagyobb projektre lényegesen gyorsabban fut le, kevesebb erőforrást igényel, mint előtte.
JSAn2Lim sikeresen alakítja át a JSAN outputot lim formátumba, typescriptes elemzések során is.

\noindent\textbf{Kulcsszavak:}

JavaScript, TypeScript, Optimalizálás, Visual Paradigm, C++
