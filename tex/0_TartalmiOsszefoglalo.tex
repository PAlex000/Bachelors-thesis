\chapter*{Tartalmi összefoglaló}
\addcontentsline{toc}{section}{Tartalmi összefoglaló}

\noindent\textbf{A téma megnevezése:}

SourceMeter JavaScript kiegészítése TypeScript támogatással, az eszközcsalád egyes eszközeinek optimalizálása.

\noindent\textbf{A megadott feladat megfogalmazása:}

A feladat során el kell érni azt, hogy a SourceMeter for JavaScript elemző eszközkészlet képes legyen jól lefutni és elemezni typescript fájlokra.
Továbbá, a JavaScript elemző (JavaScript Analyzer, továbbiakban JSAN) optimalizálása.

\noindent\textbf{A megoldási mód:}

A megoldás során változtatni kell a nyelvi sémán, amire az eszközök épülnek. A létrejövő változtatásokat minden eszközön elvégezni, tesztekkel bővíteni.

\noindent\textbf{Alkalmazott eszközök, módszerek:}

A megoldáshoz Visual Studio Code-t, a nyelvi séma szerkesztéséhez Visual Paradigm-t használtam.
A projekt lebuildeléséhez és ellenőrzéséhez Windowson Visual Studio 2017, Linuxon make programot használtam.

\noindent\textbf{Elért eredmények:}

A SourceMeter JavaScript képes TypeScript forráskódokat, fájlokat nagy pontossággal elemezni, nagyobb projektre lényegesen gyorsabban fut le, kevesebb erőforrást igényel.
A JSAN2Lim sikeresen alakítja át a JSAN eredményét nyelvfüggetlen modellre, TypeScript fájlok elemzése során is, amin dolgozik több eszközcsalád.

\noindent\textbf{Kulcsszavak:}

JavaScript, TypeScript, Optimalizálás, Visual Paradigm, C++, Statikus kódelemzés, AST
