\chapter{Általam változtatott programok}\label{chap:altalam_valtoztatott_programok}

\section{JavaScriptAddon változások}

\noindent


A séma változtatás után, magát a NodeAddonGenerator fájlt nem módosítottam.
Ennek ellenére a JavaScriptAddon mérete lényegesen megnőtt, mivel a séma nagy bővítésen esett át.
A JavaScript-es nyelvi elemek mellett a TypeScript-es nyelvi elemek is megtalálhatóak benne.
Emiatt sokkal több node, és hozzájuk a wrapper-ek generálódtak le.
A JavaScript-es nyelvi elemek megmaradtak, néhány helyen még javítottunk hibákat is.

\section{JSAN2Lim Bővítések}

\noindent

A JSAN tud már TypeScript fájlokat és projekteket is elemezni, emiatt a JSAN2Lim-et is fejleszteni kellett.
A JSAN2Lim használja a JavaScript és a Lim asg fájlokat egyaránt.

Több helyen is át kellett írni a JSAN2Lim-et:
\begin{itemize}
      \item Több visitor-nál a várt paraméter típusát átírni, például Class-ról ClassDeclarationBase-re, mivel a sémában már nem Class-ként, hanem ClassDeclarationBase-ként szerepel.
      \item Több visitor-nál a várt paraméternél helyét meg kellett változtatni egyes node-nál.
      Például a RestElement eddig a Statement package-en belül helyezkedett el, de a bővítés során átkerült a Parameter package-be.
      \item A Pattern, mint node, Parameter-re lett átnevezve. A helye is megváltozott, Statement package-ből átkerült a Parameter package-be.
      \item Típus lekérdezésnél a \texttt{getIsClass} helyett \texttt{getIsClassDeclarationBase} kellett használni.
\end{itemize}

A bővítések, hogy mikkel bővítettem a JSAN2Lim-et:
\begin{itemize}
      \item Több kind-ot is létrehoztunk a sémában, ezekkel bővíteni kellett a JSAN2Lim-et.
      \item Új node visitor-ok írása. Mint például a TSEnumDeclaration, TSImportEqualsDeclaration, TSInterFaceDeclaration, és még TypeScript node amihez kell visitor.
      A visitor-hoz a fillData függvényeket is megírni.
      \item JSAN2Lim-ben a kindStrings tömb kiegészítése TypeScript-es node-okkal.
      \item Hibák kijavítása, a VariableDeclaration-nél nem detektálta az összes változó deklarálást, főleg metódusokon belül.
      \item A \texttt{getLimKind} metódus bővítése \texttt{getIsTSTypeAliasDeclaration}, \texttt{getIsTSInterfaceDeclaration},
      \texttt{getIsTSAbstractMethodDefinition} és \texttt{getIsTSAbstractPropertyDefinition} esetekkel.
      \item TSEmptyBodyFunctionExpression osztály hibajavítása, hiszen több helyen is leállt a program emiatt.
\end{itemize}

\section{Ast binder optimalizálás}

\noindent

Az AST binder referencia kötésekre (továbbiakban: bindolásra) alkalmas. Ezek a referenciák a JSAN2Lim programhoz szükségesek.
Binder néven van definiálva a metódus a JSAN-ban.
Kétszer van használva, ezért is kulcsfontosságú, hogy gyors legyen.
Egyszer VariableUsages referenciákat bindol és egyszer ACG referenciákat bindol.

\noindent

A binder 4 argumentumot vár:
\todoi{ACG megnézése, hogy mit rövidít}
\begin{itemize}
      \item Egy stringet, ami lehet vagy VariableUsages (továbbiakban: VU) vagy ACG (egy javascript callgraph).
      \item Abstract Syntax Tree-t (továbbiakban: AST), ami egyedien van felépítve.
      \item Egy tömböt, amiben JSON objektumok találhatóak, amik a linkeket tartalmazzák.
      \item Végül még egy stringet, ami lehet addCalls, vagy setRefersTo. Az AddCalls és a SetRefersTo a JavaScriptAddon-ban található meg és onnan hívódik meg.
      Abban az esetben kap addCalls értéket, ha ACG referenciákat bindol a binder, és akkor setRefersTo, ha VU referenciákat.
\end{itemize}

A linkeket tartalmó tömb egy JSON objektuma a következőképpen néz ki:

\begin{lstlisting}[caption={Binder JSON objektuma}, label={lst:binder_json_arg}, language={JavaScript}]
source: {
      label: IdentifierNeve,
      file: AbsPath,
      start: { row: <int>, column: <int>},
      end: { row: <int>, column: <int>},
      range: { start: <int>, end: <int>},
      node: [Object]
},
target: {...}
\end{lstlisting}

Ilyen JSON elemekből épül fel a tömb.
A source és a target objektum felépítése ugyanaz, csak másak az értékek.
A kódrészletben a source van kifejtve bővebben.
A \texttt{label} az egy Identifier vagy PrivateIdentifier node nevét fogja jelölni.
A \texttt{file} egy abszolút útvonalat kap, ami az Identifiert tartalmazó fájlt jelöli.
A \texttt{start} az egy objektum, amiben külön van sor és oszlop meghatározva. Ez az Identifier első karakter pozíciójának a sor és oszlop értékei.
Az \texttt{end} ugyanaz, mint a \texttt{start}, csak itt az utolsó karakter pozíciójának a sor és oszlop értékei.
A \texttt{range} is egy objektum, a start ennél a kezdő karakter hanyadik karakter volt a kódban, és az end pedig az Identifier utolsó karakterének a karakterszáma.
A \texttt{node} is egy objektum ami maga az Identifier vagy PrivateIdentifier.

\subsection{Lassúság okai}

Binder-nek a futásideje lényegesen megnő, ha nagyonn projektekre futtatjuk le.
Ennek több oka is van:

\begin{itemize}
      \item Minden node hívásnál meghívja a JavaScriptAddon-ból egy metódust.
      A JavaScriptAddon már magában nagy terjedelmű fájl. TypeScript támogatás után kétszeresére nőtt a mérete, mint ezelőtt, emiatt lassult is.
      \item Nagyobb projektekben több függvényhívás és változószám található az elemzett kódban.
      \item Az AST nagyobb projekteknél nagyon nagy is lehet.
      Ezt az AST-t bejárjuk többször is bindolás alatt. Emiatt lesz nagyon lassú a binder.
      \item A JSON objektumokat tartalmazó tömb is nagy méretű lesz, de ez kevésbé lassítja a bindert, mint az AST-s bejárás.
\end{itemize}

\subsection{Optimalizálás}

\noindent

Optimalizálás során először a JavaScript Profiler-t használtam memória és futásidő vizsgálatra.
Nekem ez nem felelt meg, mivel a JavaScriptAddon-ban rengeteg minden egymásra épül, és 50-100 mélységű függvényhívásnál nem tudta kiírni a futásidejét és a memórihasználatot.
Ezután a binder metódust 3 részre bontottam \texttt{console.time} és \texttt{console.timeEnd} beépített függvénnyek segítségével.
Futásidőt írta ki a függvény a \texttt{console.time} és a \texttt{console.timeEnd} sorok között milliszekundumban.
Kiderült, hogy a következő sor felettébb nagy futási idővel rendelkezik:

\begin{lstlisting}[caption={Lassú metódus}, label={lst:binder_problemas_function}, language={JavaScript}]
getWrapperOfNode(resolveNode(astSet, sourceFile, element.source.range.start, element.source.range.end, true));
\end{lstlisting}

A resolveNode metódus első paraméterként vár egy AST-t, utána egy fájlnevet, kezdő- és végPozíciót, illetve egy igaz vagy hamis értéket.

Az AST-t forEach-el bejárja a program. Az AST egyes elemei az AST node-ok.
A forEach-en belül az AST node-on walk metódus segítségével bejártuk, addig amíg nem találtuk meg a keresett node-ot.
Rosszabb esetben az AST legvégén volt a keresett node, mivel már a végén voltunk a bindolásnak.
Kisebb projektek esetén ez nem baj, mivel az AST mérete nem akkora, mint egy nagyobb projekt esetében.

\noindent

Optimalizálás során, létrehoztam indexAST metódust, ami indexeli az AST-t.

\begin{lstlisting}[caption={indexAST metódus}, label={lst:indexAST_function}, language={JavaScript}]
let indexAST = function (ast) {
      ast.forEach(astNode =>{
            globals.setActualFile(astNode.filename)
            walk(astNode, {
                  enter: function(node){
                  globals.setIndexed(astNode.filename, node.range[0], node.range[1], node)}})})
      return globals.indexedAST}
\end{lstlisting}

\Aref{lst:indexAST_function} kódrészletben látható, hogy egy AST-t várunk paraméterben. Ezt a bindertől fogja kapni.
A metódusban forEach-el bejárjuk az AST elemeit, amik az AST node-ok.
Beállítom a \texttt{globals.setActualFile()} függvénnyel az aktuális fájlnevet.
Ezután az adott AST node-ot walk függvény segítségével bejárjuk.
Csak az enter metódusát kellett szerkeszteni.
Ezután a setIndexed metódus meghívódik.
A setIndexed függvénynek megkapja a fájlnevet, a node range-nek a kezdő- és a végparaméterét, ahol kezdődik az adott node és hol végződik, karakterpontosan, és magát a node-ot.

\begin{lstlisting}[caption={setIndexed metódus}, label={lst:setIndexed_function}, language={JavaScript}]
const setIndexed = function(filename, range_start, range_end, node){
      let actualfilename = getFilePathAlt(filename)
      if (indexedAST[actualfilename + "-" + range_start + "-" + range_end] !== undefined && indexedAST[actualfilename + "-" + range_start + "-" + range_end] !== node){
            return
      }
      indexedAST[actualfilename + "-" + range_start + "-" + range_end] = node
}
\end{lstlisting}

A setIndexed először vizsgál arra, hogy az adott node be van-e már indexelve.
Ezt úgy teszi meg, hogy az indexedAST tömbben keres egy indexre.
Ez az index a következőképp néz ki: FájlNév-KezdőPozíció-VégPozíció.
Továbbá vizsgál arra is, hogy ha van ilyen indexű elem a tömbben, akkor ezen az indexen található-e már ilyen node.
Ha van akkor nem állít be semmit, csak kilép, mivel már be van indexelve az adott node. Ha nincs, akkor beállítja az indexedAST tömbnek az adott indexre az adott node-ot.
Az AST-t csak egyszer járjuk be a foreach-el. A bejárás után minden node be lesz indexelve a tömbbe.

\noindent

\Aref{lst:binder_problemas_function} kódrészleten látható, hogy először a resolveNode függvény segítségével megkerestük a keresett node-ot kezdő- és végPozíció alapján.
Ezután a megkapott node-ra meghívtuk a \texttt{getWrapperOfNode} függvényt, hogy megkapjuk a wrapperjét a JavaScriptAddon-ból.
A keresést megváltoztattam, írtam rá egy getIndexed metódust.

\noindent
\todoi{Source vagy target Nodeot fordítsd le}
A \texttt{getIndexed} függvény megvizsgálja, hogy source vagy target node-ot keresünk.
Ha target node-ot, akkor visszaadjuk a \texttt{getWrapperOfNode(indexedAST[FájlNév-KezdőPozíció-VégPozíció])}-t.
Ha source node-ot keresünk, akkor a walk függvény segítségével bejárjuk a beindexelt tömböt.
Ez lényegesen gyorsabb, mint a resolveNode-os megoldás, mert ott minden egyes esetben az egész AST-t bejártuk a walk függvény segítségével, jelen esetben csak a beindexelt tömb elemét járjuk be.
Ha megkaptuk a source node-ot akkor visszaadjuk azt a \texttt{getWrapperOfNode(result)} hívással.

\subsection{Eredmények}

\todoi{Kisebb és nagyobb projekt összehasonlítása}

\begin{lstlisting}[caption={JSAN lefutási idő előtte és utána}, label={lst:jsan_before_after_comparison}]
Optimalizalt_verzio:
Binding VU: 71599/71599
VU binding: 1:39.140 (m:ss.mmm)

Optimalizalatlan_verzio:
Binding VU: 71599/71599
VU binding: 42:31.211 (m:ss.mmm)
\end{lstlisting}

Ugyanazt az outputot adja mind a kettő program, szóval csak gyorsaságban és memóriahasználatban változott sokat.

Emellett a JavaScriptSchema újraírása is sikeres volt, emiatt a JSAN tud typescriptes kódokat elemezni.
JSAN2Limet sikeresen módosítottam, hogy a JSAN által kiadott outputot sikeresen limmé alakítsa.
