\chapter{Általam változtatott programok}\label{chap:altalam_valtoztatott_programok}

\section{JavaScriptAddon változások}

\noindent

A JavaScriptSchema változtatások után, a NodeAddonGeneratort nem módosítottuk.
Ennek ellenére a javascriptAddon.node mérete megduplázódott, csupán azért, mert a schéma ekkora bővítésen esett át.
Sokkal több nodeWrapper lett, mivel bejöttek a typescript általi dolgok is a javascript mellett.
Természetesen a javascriptes dolgok megmaradtak, így javascriptet ugyanolyan jól tud elemezni, sőt néhány esetben jobban is, mert voltak figyelmetlenségek előző verzióban.

\noindent

Mivel a javascriptAddon mérete nőtt, ezáltal a sebesség csökkent.
Ezt majd egy másik fejezetben jobban kifejtem, célunk az volt, hogy először tudjon typescriptes fileokat és projekteket elemezni a jsan.


\section{JSAN2Lim}

\noindent

A JSAN2Lim egy c++-ban megírt program. A JSAN általi outputot átalakítja lim formátumú fileokra.
Azért lim formátumú fileokra, mivel a lim az egy környezetfüggetlen nyelv, és sokkal könnyebb ezen metrikákat mérni.
A JSAN2Lim is egy alprojektje az Analyzer JavaScriptnek.
Fontos, hogyha a JSAN programot fejlesztik, akkor a JSAN2Lim-et is fejleszteni kell, mivel ha nem, akkor a metrikák nem fognak jól számolódni.
A JSAN2Lim használja a LimSchemat, ez is ugyanúgy egy $.vpp$ kiterjesztésű file.
Más logikával van felépítve, mint a JavaScriptSchema, ezért nem is nagyon mennék bele az elemzésébe, csak csupán néhány dolgot fogok használni belőle.
A JSAN2Lim visiteli az összes nodeot amit a JSAN outputba kirak, ezért JSAN2LimVisitor a fő file neve.
Minden egyes nodera külön meg kell írni a visit függvényt, hogy tudja a c++ program, hogy mit csináljon ha olyan nodeot kap.

\noindent

A JSAN2Lim működését egy kódrészleten keresztül mutatom be.
\begin{lstlisting}[caption={ClassDeclaration Visitor}, label={lst:classdeclaration_visitor}, language={CStyle}]
void JSAN2LimVisitor::visit(const javascript::asg::structure::ClassDeclarationBase& clNode, bool callVirtualBase)
{
      VISIT_BEGIN(clNode, callVirtualBase);
      lim::asg::logical::Class& classLimNode = dynamic_cast<lim::asg::logical::Class&>(createLimNode(clNode, callVirtualBase));
      fillData(classLimNode, clNode);//sets the name
      fillMemberData(classLimNode);
      demangledNameParts.push_back(classLimNode.getName());
      classStack.push(ClassInfo());
      classStack.top().classNodeId = classLimNode.getId();
      if (clNode.getSuperClass() != NULL) {
      if (javascript::asg::Common::getIsIdentifier(*clNode.getSuperClass())) {
      javascript::asg::expression::Identifier& superClass = dynamic_cast<javascript::asg::expression::Identifier&>(*clNode.getSuperClass());
      classStack.top().isSubclass = superClass.getId();
      }
      }
      classStack.top().TLOC = clNode.getPosition().getEndLine() - clNode.getPosition().getLine() + 1;
      classStack.top().LOC = clNode.getPosition().getEndLine() - clNode.getPosition().getLine() + 1;
      if (!packageStack.empty()) {
      lim::asg::logical::Package& packageLimNode = dynamic_cast<lim::asg::logical::Package&>(limFactory.getRef(packageStack.top().packageNodeId));
      SAFE_EDGE(packageLimNode, Member, lim::asg::logical::Member, classLimNode.getId());
      }
      else {
      common::WriteMsg::write(common::WriteMsg::mlDebug, "Empty packageStack while visiting node %d.", clNode.getId());
      }
      addIsContainedInEdge(classLimNode, clNode);
}
\end{lstlisting}

A VISIT\_BEGIN definiálva van a file elején, ami a ViscitorAbstractNodesnak a visit metódusa.
Ezután létrehozok egy classLimNode nevű változót, ami ${lim::asg::logical::Class}$ típusú, ez a limFactoryba található meg.
Ennek a változónak adunk egy értéket, ami a createLimNode által visszaadott node lesz, és ezt castoljuk erre a típusra.

\noindent

A createLimNode meghívja a limFactoryból a createNode-ot, ami vár paraméterbe egy nodeKind-ot. Ezt a nodeKind-ot mi adjuk meg neki, az alapján, hogy az adott node milyen típusú.
A nodeKindot a getLimKind függvény határozza meg a következők szerint.
Megvizsgáljuk, hogy az adott node $getIsClassDeclarationBase()$ vagy $getIsMethodDefinition$ vagy $getIsFunctionDeclarationBase$ vagy sorolhatnám még. Ebből van 10-15db.
Típustól szerint adjuk meg a nodeKindnak az értékeket, amik lehetnek a következők: ndkClass, ndkMethod, ndkParameter, ndkMethodCall, ndkAttribute, ndkComment, ndkPackage, ndkAttributeAccess, ndkComponent.
Ezek mind a lim-nek az asgben való típusok és mindet tudja kezelni. Miután beállítottam a nodeKind értékét, szimplán returnolom.
Ha megkaptuk a nodeKind értékét, akkor ezt létrehozza a createLimNode függvény. Állít be ennek pozíciót és returnoli a *limNode-ot.

\noindent

Ezáltal most megkaptuk a classLimNode-ot, hiszen a nodeKind az ndkClass lett és ennek megfelelően hozta létre a lim.
Ezután meghívjuk a filldata függvényt, első paraméternek magát a létrehozott classLimNodeot, másodiknak magát a nodeot adjuk meg.
A JSAN által kiadott outputot $.jssi$ kiterjesztésű fileon végig megy. Visitorok segítségével oldja meg ezt.
Ebben a filldatában beállítom a limnode láthatóságát, nyelvét (ami konstans javascript), azt, hogy abstract-e.
Lekérem a nodeIdját és a lastLimMemberNodeId illetve a lastLimScopeNodeIdba pusholom. Ezután beállítom a classKindját, ami clkClass lesz (ezt is a factoryból).
Aztán a nevét szeretném beállítani, a node-nak lekérem azt,hogy identifier-e, vagy sem. Ha nem null, akkor lekérem az identifiert és $identifier.getName()$el lekérem a nevét és be is állítom a limNodenak a $setName()$ függvénnyel.
Ha az identifier null, akkor a $limNode.setIsAnonymous()$-t beállítom true értékre és adok neki egy $"anonymousClass"+ ss.str()$ ahol az ss az egy stringstream, szimplán egy counter, hogy eddig hány anonymous class volt.
Ezután megvizsgálom, hogy e methodstack empty-e, ha nem, akkor van szülője, ezért a SAFE\_EDGE function hozzáadok egy eegy edget a parent és a jelenlegi limnode-hoz.
Ezt ugyanúgy megvizsgálom packageStackre, ha nem empty, akkor ugyanezt elvégzem csak a parent most más lesz.
Végezetül az $addIsContainedInEdge$ függvényt meghívom a limnode és a jsNode paraméterekkel.

\noindent

Ezután a $fillMemberData()$ függvényt hívjuk meg, ami igazából csak pozíciót állít be, illetve a neven picit változtat.
Utána a classStack-hez hozzáadja az adott osztály tulajdonságait, beállítja a parent classt ha van, pozíciót állít be, és a végén ugyanúgy meghívjuk a $addIsContainedInEdge$ a classLimNode és a clNodeal.
Ez a procedúra történik minden egyes nodenál a JSAN2Limben.

\subsection{Bővítések}

\noindent

Mivel a JSAN mostmár tud typescriptet is elemezni, emiatt a JSAN2Lim-et is fel kell erre készíteni, hogyha typescriptes nodeokat kap, akkor tudja, hogy mit kell azzal csinálni.
A javascript asg-t is használja a JSAN2Lim, ezért sok helyen először változtatni kellett néhány dolgon.
\begin{itemize}
      \item Néhány visitornál a várt paraméter típusát átírni, például Classról ClassDeclarationBase-re, hiszen a schémában változtatás volt.
      \item Néhány visitornál a várt paraméternél a helyét az adott nodenak. Például a RestElement már nem a statement packagen belül van, hanem a parameteren belül.
      \item A Pattern mint node megszűnt, Parameterre lett átnevezve, és a helye is megváltozott, statementből átkerült parameter packagebe
      \item Típus lekérdezésnél a $getIsClass()$ helyett például $getIsClassDeclarationBase()$ lett, változott ez is a JavaScriptSchema miatt.
\end{itemize}

A bővítések, hogy miket adtam hozzá a JSAN2LImhez:
\begin{itemize}
      \item Az új Kindokat amiket felvettünk a JavaScriptSchemaban. A következőképp: $std::string getEnumText(javascript::asg::ASTNodeTypes);$
      \item Új visitorok írása. TSEnumDeclaration, TSImportEqualsDeclaration, TSInterFaceDeclaration, és még sok másik, ezekhez a filldatákat is megírni
      \item kindStrings tömb kiegészítése a typescriptes nodeokkal
      \item Bugok kijavítása, a VariableDeclarationnél nem vette észre az összes declarationt, főleg metóduson belül.
      \item A $getLimKind()$ function bővítése $getIsTSTypeAliasDeclaration()$, $getIsTSInterfaceDeclaration()$, $getIsTSAbstractMethodDefinition()$ és $getIsTSAbstractPropertyDefinition()$ else if ágakkal.
      \item TSEmptyBodyFunctionExpression debugolása, sok helyen lehalt a program emiatt.
\end{itemize}


\section{Ast binder optimalizálás}

Az AST binder referenciákat bindol. Ezek a referenciák a JSAN2Lim programhoz kellenek.
Binder néven van definiálva az astTransformer.js fileban.
Kétszer van használva, ezért is kulcsfontosságú, hogy gyors legyen.
Egyszer VariableUsages referenciákat bindol és egyszer ACG referenciákat bindol.

\noindent

A binder 4 argumentumot vár:

\begin{itemize}
      \item Egy stringet, ami lehet vagy VU (ami VariableUsages-t rövidíti) vagy ACG (egy javascript callgraph).
      \item Egy Abstract Syntax Tree-t (AST), ami egyedien van felépítve.
      \item Egy tömböt, amiben JSON objektumok találhatóak, amik a linkeket tartalmazzák.
      \item Végül még egy stringet, ami lehet addCalls, vagy setRefersTo. Az AddCalls és a SetRefersTo a javascriptAddonban található meg és onnan hívódik meg.
      Akkor addCalls ha ACG referenciákat bindolunk és akkor setRefersTo ha VU referenciákat.
\end{itemize}

A linkeket tartalmó tömb egyik JSON objektuma a következőképp néz ki:

\begin{lstlisting}[caption={Binder JSON argumentuma}, label={lst:binder_json_arg}, language={JavaScript}]
source: {
      label: IdentifierNeve,
      file: AbsPath,
      start: { row: <int>, column: <int>},
      end: { row: <int>, column: <int>},
      range: { start: <int>, end: <int>},
      node: [Object]
},
target: {...}
\end{lstlisting}

\Aref{lst:binder_json_arg} kódrészletben egy JSON elemét láthatjuk a tömbnek. Ilyen JSON elemekből épül fel a tömb.
A source és a target felépítése ugyanaz, csupán másak az értékek bennük.
A kódrészletben csak a sourcet fejtettem ki kicsit bővebben. A label az egy Identifier vagy PrivateIdentifier nevét fogja jelölni.
A file egy abszolút útvonalat, ami az Identifiert tartalmazó filet jelöli.
A start az egy object, amiben külön van sor és oszlop meghatározva. Ez az Identifier első karakter pozíciójának a sor és oszlop értékei.
Az end ugyanaz, mint a start, csak itt az utolsó karakter pozíciójának a sor és oszlop értékei.
A range is egy object, a start ennél a kezdő karakter hanyadik karakter volt a kódban, és az end pedig az identifier utolsó karakterének a karakterszáma.
A node is egy object ami maga az Identifier vagy PrivateIdentifier.

\subsection{Lassúság okai}

A binder nagyobb projektekre lassan fut le. Ennek több oka is van:

\begin{itemize}
      \item Minden hívásnál meghívja a javascriptAddon.nodeot, ami már magában nagy file. TypeScript támogatás után kétszeres lett a mérete, ezáltal sokkal lassabb lett.
      \item Mivel nagyobb a projekt, ezért sokkal több a függvényhívás is az elemzett kódban, emellett sok a változószám is.
      \item Az AST nagyon nagy, és ezt bejárjuk többször is bindolás alatt. Emiatt nagyon lassú a program nagy projektek esetén.
      \item A JSONokat tartalmazó tömb is nagyon nagy lesz, de annyira ez nem lassítja le a programot.
\end{itemize}

\subsection{Optimalizálás}

\noindent

Először ki kellett derítenem, pontosan melyik funkció mennyi memóriát vesz igénybe és milyen gyorsan fut le.
Erre találtam egy javascript profilert, ami kiírta minden függvényhívásnál a lefoglalt memóriát és a memória használatot.
Ez nekem nem volt a legjobb, mivel nagyon egybe van ágyazva minden, és eléggé mélyre mentek a functionok.
Ezután próbáltam picit egyszerűbbet, a binder kódját 3 részre osztani, ${console.time()}$ és ${console.timeEnd()}$ használatával.
Ez a ${console.time()}$ parancs kiírja nekem ms-ben, hogy mennyi idő telt el amíg eljutott a ${console.timeEnd()}$ig.
Így megkaptam, hogy melyik rész nagyjából mennyi idő alatt fut le, könnyebb volt szűkíteni, hogy melyik function lesz a bajos.
Végül kiderült, hogy a következő sor lassítja be nagyon a programot:

\begin{lstlisting}[caption={Problémás function}, label={lst:binder_problemas_function}, language={JavaScript}]
globals.getWrapperOfNode(resolveNode(astSet, sourceFile, element.source.range.start, element.source.range.end, true));
\end{lstlisting}

Ezen belül is a resolveNode function volt a probléma.
A resolveNode kapott egy ast-t (amit a binder kapott meg paraméterbe), egy filenevet, kezdő- és végPozíciót, illetve egy true vagy false értéket, ami attól függ, hogy sourcenodeot vagy targetnodeot keresünk.
Az ast-n forEach-el végigmentünk, amin belül az astNodeon walkoltunk addig amíg nem találtuk meg a számunkra megfelelő nodeot.
Rosszabb esetben a legvégén volt a node, mivel már a végén voltunk a bindolásnak, és ebből is látható volt, hogy ez kellően sok időbe is kerülhet, ha rengeteg nodeunk van.
Kisebb projekt esetében ez nem baj, mivel az ast abban az esetben nem annyira nagy.
Ahol már van 100 vagy 200 ezer változó és vagy függvényhívás, ott már nagyobb a baj, mivel ezt rengetegszer meg kell ismételni.
Ezután ötleteltem, hogy hogyan tudnám jobbra átírni a resolveNodeot functiont, vagy egy másik logika alapján megközelíteni a problémát.
Végül eszembe jutott egy sokkal könnyebb megoldás erre, ami nem igényli a sokszori bejárását az astnek.
\noindent

Először is, létrehoztam egy indexAST nevű functiont, ami a következőket hajtja végre:

\begin{lstlisting}[caption={indexAST function}, label={lst:indexAST_function}, language={JavaScript}]
let indexAST = function (ast) {
      ast.forEach(astNode =>{
            globals.setActualFile(astNode.filename)
            walk(astNode, {
                  enter: function(node){
                  globals.setIndexed(astNode.filename, node.range[0], node.range[1], node)}})})
      return globals.indexedAST}
\end{lstlisting}

\Aref{lst:indexAST_function} kódrészletben látható, hogy egy ast-t várunk paraméterben. Ezt a bindertől fogja kapni, ez az egyénilég létrehozott ast.
A functionben forEach-el végig megyünk az ast elemein, amik az astNodeok. Itt beállítjuk a ${globals.setActualFile()}$ függvénnyel az aktuális filenevet.
Ezután az adott astNode-on elkezdünk walkolni. Csak az enter metódusát írtam meg. Meghívunk egy függvényt ami a globalsban lett definiálva, setIndexed a neve.
Ennek a függvénynek megadjuk a filevenet, a node rangenek a kezdő- és a végparaméterét (tehát ahol kezdődik az adott node és hol végződik, karakterpontosan), és magát a nodeot.

\begin{lstlisting}[caption={setIndexed function}, label={lst:setIndexed_function}, language={JavaScript}]
const setIndexed = function(filename, range_start, range_end, node){
      let actualfilename = getFilePathAlt(filename)
      if (indexedAST[actualfilename + "-" + range_start + "-" + range_end] !== undefined && indexedAST[actualfilename + "-" + range_start + "-" + range_end] !== node){
            return
      }
      indexedAST[actualfilename + "-" + range_start + "-" + range_end] = node
}
\end{lstlisting}

A setIndexed először végez egy vizsgálatot arra, hogy az adott node be van már indexelve.
Ezt úgy teszi meg, hogy az indexedAST tömbben keres egy indexre (Ez az index a következőképp néz ki: fileNev-StartPosition-Endposition), és még vizsgál is arra, hogy ha van ilyen index, akkor ezen van emár olyan node.
Ha van akkor nem állít be semmit, csak returnol, mivel már be van indexelve az adott node. Ha nincs, akkor beállítja az indexedAST tömbnek az adott indexre az adott nodeot.

\noindent

Ezáltal az adott astn csupán csak egyszer megyünk végig foreach-el és egyszer walkolunk, ekkor egy tömbbe beindelexünk minden egyes létező nodeot ami kellhet nekünk.
Ha ezzel megvoltunk akkor utána kezdődhet a binder része.
\Aref{lst:binder_problemas_function} látható, hogy először megkerestük a nodeot index alapján és aztán hívtuk meg rá a ${getWrapperOfNode()}$ függvényt.
Ezt is megváltoztattam, írtam rá egy getIndexed függvényt.
A ${getIndexed()}$ függvény megvizsgálja, hogy source vagy targetNodeot keresünk. Ha targetNodeot, akkor returnoljuk a ${getWrapperOfNode(indexedAST[filename-StartPosition-Endposition])}$-t.
Ha targetNodeot keresünk, akkor walkolunk ismét, de nem az ast-ben, hanem már a beindexelt tömbben.
Ez lényegesen gyorsabb, mint a resolveNodeos megoldás, mert ott minden egyes esetben az ast-n walkoltunk, itt meg csak egy beindexelt elemén a tömbnek.
Ha megkaptuk a sourceNodeot akkor returnoljuk a ${getWrapperOfNode(result)}$t.
