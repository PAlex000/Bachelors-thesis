\chapter{Bevezetés}

\noindent

Szakdolgozatomban a SourceMeter for JavaScript továbbfejlesztése volt a cél.
A JavaScript elemzése mellett TypeScript nyelvű fájlokat és projekteket is elemezni kellett.
Ebből adódik az, hogy gyökerestül át kellett írni az eszközt, annak érdekében,
hogy JavaScriptet és TypeScriptet is tudjon egyaránt elemezni.

\noindent

A SourceMeter for JavaScript projekten többen is fejlesztettek egyszerre, ezáltal voltak átfedések, oszthatatlan részek.
A projekt néhány része is egymásra épül, ezért kiemelten fontos volt a csapatmunka egyes részeknél, hogy a projekt eredményesen és hatékonyan haladjon.
Az eredményes csapatmunka magában foglalja az eredmények és az előrehaladás folyamatos kommunikálását,
ami azt jelenti, hogy mindketten megértjük, hogy milyen célkitűzések vannak a projektben,
és rendszeresen jelentést adunk egymásnak az elvégzett munkáról és az elért eredményekről.

\noindent

Megtalálható számtalan JavaScript és TypeScript fájl elemző eszköz amiknek nyílt a forráskódja.
Ezek között megtalálható a Codehawk CLI, Codelyzer vagy a CodeClimmate-Duplication.
Mind a három eszköz jól tud elemezni JavaScript és TypeScript fájlokat, viszont csak egy-egy specifikus esetre jók.

\noindent

Fejlesztés során a csoportos munka elkerülhetetlen volt, hiszen volt olyan rész, amire számtalan eszköz épült.
Ez a JavaScript nyelvi séma (továbbiakban:séma) szerkesztése volt. Ezt \aref{chap:nyelvi_sema} alfejezetben taglalom bővebben.

\noindent

A séma szerkesztése után a JavaScript Analyzer To Lim (továbbiakban JSAN2Lim) továbbfejlesztése volt a feladatom.
A továbbfejlesztés C++, C nyelv, TypeScript nyelvi elemek$^{~\cite{fenton2014pro, cherny2019programming, nance2014typescript, 10.1007/978-3-662-44202-9_11}}$ és a TypeScript séma$^{~\cite{typescript-eslint}}$ ismeretét igényelte meg.
Emiatt a fejlesztés előtt tanulmányozni kellett ezeket.

\noindent

A JSAN2Lim fejlesztése után a JavaScript elemzőnek (JavaScript Analyzer, továbbiakban: JSAN) a binder függvényét kellett optimalizálnom.
Először megérteni kellett a kódot, át kellett nézni az Absztrakt Szintaxis Fát (Abstract Syntax Tree, továbbiakban: AST), illetve több adatszerkezetet is, gyorsasági szempontból.

\noindent

Legvégül bővítettem a regressziós teszteket új projektek behozatalával. Ezt \aref{chap:Regteszt_frissítés} fejezetben taglalom.
