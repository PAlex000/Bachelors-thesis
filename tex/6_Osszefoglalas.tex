\chapter{Összefoglaló}\label{chap:Összefoglaló}

\section{Program javulása}

\noindent

A szakdolgozatom során sikerült elérnem azt, hogy az Ast binder közel tízszer gyorsabban fut le nagyobb projektre, mint ezelőtt.
Leteszteltem ugyanarra a nagy projektre az eredeti JSAN lefutását és az én általam átírt lefutását. Ezt \Aref{lst:jsan_before_after_comparison} kódrészleten láthatjuk.
Kisebb projektekre körülbelül ötszörös gyorsulás van.
\begin{lstlisting}[caption={JSAN lefutási idő előtte és utána}, label={lst:jsan_before_after_comparison}]
PlaceHolder
\end{lstlisting}

Emellett a JavaScriptSchema újraírása is sikeres volt, emiatt a JSAN tud typescriptes kódokat elemezni.
JSAN2Limet sikeresen módosítottam, hogy a JSAN által kiadott outputot sikeresen limmé alakítsa.
\section{Jövőbeli tervek}

\noindent

JavaScriptSchema bővítése sok időbe telt, mivel se dokumentáció, se tapasztalat nem volt. Ezután még a JSAN2Lim átírás is sok időbe került.
Eközben a typescript-eslint github repo amit használtunk a bővítésre, frissült, sok új funkciót hoztak be.
Rengeteg változtatás volt a typescript résznél, de a schémát nem tudtuk még naprakészre hozni, mivel voltak ennél fontosabb feladatok.
Egyik jövőbeli terv az, hogy a schémát up to datere hozzam, mivel már van hozzá dokumentáció is, meg nagyjából én is írtam, ezért nem lesz ez annyi idő, mint volt az elején az újraírása.

\noindent

Végül a JSAN programra még bőven ráfér az optimalizálás, mivel csak az ast bindert optimalizáltuk, rengeteg helyen feleslegesen van bejárva az ast.
Ez a másik jövőbeli terv.
