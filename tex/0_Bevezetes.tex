\chapter*{Bevezetés}
\addcontentsline{toc}{chapter}{Bevezetés}

\noindent

Szakdolgozatomban a JavaScript Analyzer továbbfejlesztése a cél.
A továbbfejlesztés arról szól, hogy a javascript mellett előtérbe jön a typescript is.
Ezáltal a typescript fileokat és projekteket is elemezni kell.
Ez azt eredményezi, hogy a Javascript Analyzert gyökerestül át kell írni, annak érdekében,
hogy JavaScriptet és TypeScriptet is tudjon egyaránt elemezni.

\noindent

A JavaScript Analyzer az egy nagyobb projektnek az egyik alprojektje. Ezt a nagyobb projektet Analyzer-JavaScriptnek hívják.
Az Analyzer-Javascriptnek több alprojektje is van, amiben sok minden egymásra épül.
A szakdolgozatomban a következő alprojektekben fogok változtatni és bővíteni: JavaScriptSchema, JSAN (ezen belül is csak az astTransformerben), és az erre épülő JSAN2Lim.
Emellett még megtalálhatóak a következő alprojektek is: ESLintRunner, ESLint2Graph, LIM2Metrics, LIM2Patterns, DuplicatedCodeFinder és a ChangeTracker.
Azért kell a JSAN2Limben is változtatni, mivel ha gyökerestül megváltoztatom a JSAN-t,
akkor a JSAN2Lim rossz eredményeket fog visszaadni a typescript fileok elemzése közben.

\noindent

A projekten amin fejlesztettem, többen is fejlesztettek egyszerre, ezáltal voltak átfedések, oszthatatlan részek.
A projekt néhány része egymásra épül, ezért kiemelten fontos volt a csapatmunka egyes részeknél, hogy a projekt eredményesen és hatékonyan haladjon.
Az eredményes csapatmunka magában foglalja az eredmények és az előrehaladás folyamatos kommunikálását,
ami azt jelenti, hogy mindketten megértjük, hogy milyen célkitűzések vannak a projektben,
és rendszeresen jelentést adunk egymásnak az elvégzett munkáról és az elért eredményekről.

\noindent

Fejlesztés során JavaScriptSchema file szerkesztése csoportos munka volt, hiszen minden erre épült, ez volt az alapja mindennek.
Sok időt vett igénybe a szerkesztés. Nulláról kellett újraírni az egész vpp filet.
Sajnos dokumentáció nem készült az előző filehoz, ezért nekünk kellett kitalálni, hogy mi mit csinált, hiszen akik ezt írták, ők már nem foglalkoztak ezzel és elérhetetlenek voltak.
A szerkesztéshez szükséges volt a Visual Paradigm alkalmazást használni.
Ezzel egyikőnk sem találkozott még, szóval először ezt kellett tanulmányozni, megérteni.
Ezután értelmezni kellett a meglévő schemát, hiszen eddig az jól működött, csak nem lehetett könnyen bővíteni.
Mérlegeltük a két opciót, ahol vagy megpróbáljuk bővíteni a jelenlegi schémát, vagy nulláról elkezdjük újraírni.
Végül az újraírás mellett döntöttünk, hiszen ezt láttuk gyorsabb és könnyebb megoldásnak. Egy könnyen bővíthető, dokumentál schema volt az elképzelés.
Ketten fejlesztettük le végül ezt a schemát.

\noindent

Ezután változtattam a JSAN2Lim-et, hogy képes legyen értelmezni a JSAN outputjait a typescriptes elemzések során.
Itt át kellett néznem a c++ programozási nyelvet, hiszen az egész program ebben íródott, ugyanúgy dokumentáció nélkül.
Több nehézségbe is ütköztem, de a legvégén sikerült olyan állapotra hozni a programot, hogy az output nagy százalékát sikeresen átkonvertálja lim fileba.

\noindent

Ezután a JSANnak az astTransformer filejában a bindert optimalizáltam.
Értelmezni kellett a kódot, át kellett néznem az ast-t, meg több adatszerkezetet is, gyorsasági szempontból.

\noindent

Legvégül pedig miután minden készen volt, bővítettem a regteszteket új projektek behozatalával, mind javascripteseket és typescripteseket.
Ezeknek az outputjait megnéztem, hogy jó eredményt ad-e, itt még volt egy kis bugfixing a JSAN2Limben és a Schémában.
